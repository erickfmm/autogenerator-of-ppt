\documentclass[aspectratio=169]{beamer}
\usetheme{Madrid}
\usecolortheme{default}
\usepackage[utf8]{inputenc}
\usepackage[spanish]{babel}
\usepackage{amsmath}
\usepackage{amssymb}
\usepackage{tikz}
\usepackage{booktabs}

% Configuración del tema
\setbeamertemplate{navigation symbols}{}
\setbeamertemplate{footline}[frame number]

% Comandos personalizados
\newcommand{\alert}[1]{\textcolor{red}{#1}}
\newcommand{\highlight}[1]{\textcolor{blue}{#1}}

% Variables del documento (serán reemplazadas por Jinja2)
\title{Reglas de las Probabilidades}
\subtitle{Cálculo y aplicación de probabilidades}
\author{Probabilidad y Estadística}
\date{\today}

\begin{document}

% Página de título
\begin{frame}
\titlepage
\end{frame}

% Diapositivas generadas dinámicamente
\begin{frame}{Probabilidad de un Evento}
\begin{center}
\Large
$P(A) = \frac{\text{casos favorables}}{\text{casos totales}}$
\end{center}
\begin{itemize}
\item Valor entre 0 y 1 (o 0% y 100%)
\end{itemize}
\begin{itemize}
\item P(A) = 0: evento imposible
\end{itemize}
\begin{itemize}
\item P(A) = 1: evento seguro
\end{itemize}
\begin{block}{Ejemplo}
Lanzar dado: P(par) = 3/6 = 0.5
\end{block}
\end{frame}

\begin{frame}{Regla Aditiva}
\begin{itemize}
\item Probabilidad de que ocurra A o B
\end{itemize}
\begin{center}
\Large
$P(A \cup B) = P(A) + P(B) - P(A \cap B)$
\end{center}
\begin{itemize}
\item Eventos mutuamente excluyentes:
\end{itemize}
\begin{center}
\Large
$P(A \cup B) = P(A) + P(B)$
\end{center}
\begin{block}{Ejemplo}
Carta: P(As o Rey) = 4/52 + 4/52 = 8/52
\end{block}
\end{frame}

\begin{frame}{Regla Multiplicativa}
\begin{itemize}
\item Probabilidad de que ocurra A y B
\end{itemize}
\begin{center}
\Large
$P(A \cap B) = P(A) \cdot P(B|A)$
\end{center}
\begin{itemize}
\item Eventos independientes:
\end{itemize}
\begin{center}
\Large
$P(A \cap B) = P(A) \cdot P(B)$
\end{center}
\begin{block}{Ejemplo}
Dos dados: P(ambos 6) = (1/6) × (1/6) = 1/36
\end{block}
\end{frame}

\begin{frame}{Problemas Aplicados}
\begin{exampleblock}{Problema}
Una urna tiene 5 bolas rojas y 3 azules
\end{exampleblock}
\begin{itemize}
\item a) P(roja en un sorteo)
\end{itemize}
\begin{itemize}
\item b) P(roja o azul)
\end{itemize}
\begin{itemize}
\item c) P(2 rojas consecutivas sin reposición)
\end{itemize}
\begin{block}{Solución}
\begin{itemize}
\item a) P(R) = 5/8
\item b) P(R∪A) = 5/8 + 3/8 = 1
\item c) P(R∩R) = (5/8) × (4/7) = 20/56
\end{itemize}
\end{block}
\end{frame}


% Diapositiva final
\begin{frame}
\begin{center}
\Huge ¿Preguntas?
\end{center}
\end{frame}

\end{document}