\documentclass[aspectratio=169]{beamer}

% Tema moderno y lúdico
\usetheme{Boadilla}
\usecolortheme{dolphin}

% Paquetes esenciales
\usepackage[utf8]{inputenc}
\usepackage[spanish]{babel}
\usepackage{amsmath}
\usepackage{amssymb}
\usepackage{tikz}
\usetikzlibrary{shapes, arrows, positioning, calc}
\usepackage{pgfplots}
\pgfplotsset{compat=1.18}
\usepackage{pgf-pie}
\usepackage{booktabs}
\usepackage{xcolor}
\usepackage{colortbl}
\usepackage{newunicodechar}

% Configurar emojis como texto simple
\newunicodechar{🎲}{\textsf{[dado]}}
\newunicodechar{🎯}{\textsf{[objetivo]}}
\newunicodechar{📋}{\textsf{[lista]}}
\newunicodechar{⭐}{\textsf{[estrella]}}
\newunicodechar{🔮}{\textsf{[bola]}}
\newunicodechar{📏}{\textsf{[regla]}}
\newunicodechar{🃏}{\textsf{[carta]}}
\newunicodechar{🔄}{\textsf{[ciclo]}}
\newunicodechar{🔢}{\textsf{[numeros]}}
\newunicodechar{➕}{\textsf{[+]}}
\newunicodechar{✖}{\textsf{[x]}}
\newunicodechar{🔗}{\textsf{[cadena]}}
\newunicodechar{🪙}{\textsf{[moneda]}}
\newunicodechar{🔑}{\textsf{[llave]}}
\newunicodechar{🔍}{\textsf{[lupa]}}
\newunicodechar{📊}{\textsf{[grafico]}}
\newunicodechar{🎴}{\textsf{[cartas]}}
\newunicodechar{💪}{\textsf{[fuerza]}}
\newunicodechar{🌟}{\textsf{[estrella]}}
\newunicodechar{💡}{\textsf{[idea]}}
\newunicodechar{✨}{\textsf{[brillos]}}
\newunicodechar{✅}{\textsf{[check]}}
\newunicodechar{🎓}{\textsf{[gorro]}}
\newunicodechar{🤔}{\textsf{[pensar]}}
\newunicodechar{🚀}{\textsf{[cohete]}}
\newunicodechar{♥}{\ensuremath{\heartsuit}}
\newunicodechar{₁}{\textsubscript{1}}
\newunicodechar{₂}{\textsubscript{2}}
\newunicodechar{✖}{\textsf{x}}
\newunicodechar{✓}{\textsf{[ok]}}
\newunicodechar{⚫}{\textsf{[punto]}}
\newunicodechar{👥}{\textsf{[grupo]}}
\newunicodechar{️}{}

% Colores personalizados vibrantes y lúdicos
\definecolor{colorPrimario}{RGB}{41, 128, 185}      % Azul vibrante
\definecolor{colorSecundario}{RGB}{231, 76, 60}     % Rojo coral
\definecolor{colorAccento}{RGB}{46, 204, 113}       % Verde esmeralda
\definecolor{colorAdvertencia}{RGB}{241, 196, 15}   % Amarillo dorado
\definecolor{colorMorado}{RGB}{155, 89, 182}        % Morado amigable
\definecolor{colorNaranja}{RGB}{230, 126, 34}       % Naranja cálido
\definecolor{colorFondo}{RGB}{236, 240, 241}        % Gris claro de fondo

% Configuración del tema
\setbeamercolor{structure}{fg=colorPrimario}
\setbeamercolor{palette primary}{bg=colorPrimario,fg=white}
\setbeamercolor{palette secondary}{bg=colorSecundario,fg=white}
\setbeamercolor{palette tertiary}{bg=colorAccento,fg=white}
\setbeamercolor{block title}{bg=colorPrimario!90,fg=white}
\setbeamercolor{block body}{bg=colorPrimario!10,fg=black}
\setbeamercolor{block title example}{bg=colorAccento!90,fg=white}
\setbeamercolor{block body example}{bg=colorAccento!10,fg=black}
\setbeamercolor{block title alerted}{bg=colorSecundario!90,fg=white}
\setbeamercolor{block body alerted}{bg=colorSecundario!10,fg=black}

% Plantillas personalizadas
\setbeamertemplate{navigation symbols}{}
\setbeamertemplate{footline}[frame number]
\setbeamertemplate{itemize items}[circle]
\setbeamertemplate{enumerate items}[circle]
\setbeamerfont{title}{size=\huge,series=\bfseries}
\setbeamerfont{frametitle}{size=\Large,series=\bfseries}

% Comandos personalizados para énfasis
\renewcommand{\alert}[1]{\textcolor{colorSecundario}{\textbf{#1}}}
\newcommand{\highlight}[1]{\textcolor{colorPrimario}{\textbf{#1}}}
\newcommand{\importante}[1]{\textcolor{colorAdvertencia}{\textbf{#1}}}
\newcommand{\exito}[1]{\textcolor{colorAccento}{\textbf{#1}}}

% Variables del documento (serán reemplazadas por Jinja2)
\title{Reglas de las Probabilidades}
\subtitle{¡Calculando las posibilidades! }
\author{Probabilidad y Estadística}
\institute{Aprendiendo con diversión 🎓}
\date{\today}

\begin{document}

% Página de título con diseño atractivo
\begin{frame}[plain]
\begin{tikzpicture}[remember picture,overlay]
  % Fondo decorativo
  \fill[colorPrimario!20] (current page.south west) rectangle (current page.north east);
  \fill[colorAccento!30] (current page.south west) -- (current page.south east) -- 
        ([yshift=-3cm]current page.north east) -- ([yshift=-3cm]current page.north west) -- cycle;
\end{tikzpicture}
\titlepage
\end{frame}

% Diapositivas generadas dinámicamente
\begin{frame}[fragile]{¿Qué es la Probabilidad?}
\begin{itemize}
  \item Es medir qué tan posible es que algo ocurra.
\end{itemize}

\begin{alertblock}{💡 Nota importante}
¡Es como adivinar el futuro con matemáticas! 
\end{alertblock}

\begin{itemize}
  \item Va de 0 (imposible) a 1 (seguro).
\end{itemize}

\begin{itemize}
  \item También se usa en porcentajes (0% a 100%).
\end{itemize}

\begin{exampleblock}{✨ Ejemplo}
Al lanzar una moneda, ¿qué tan posible es que salga 'cara'?
\end{exampleblock}

\end{frame}

\begin{frame}[fragile]{La Escala de la Probabilidad}
\begin{itemize}
  \item 0 o 0%: Imposible (nunca, nunca pasa).
\end{itemize}

\begin{itemize}
  \item 0.25 o 25%: Poco probable (raro que pase).
\end{itemize}

\begin{itemize}
  \item 0.5 o 50%: Chance 50/50 (puede ser que sí, puede ser que no).
\end{itemize}

\begin{itemize}
  \item 0.75 o 75%: Muy probable (casi siempre pasa).
\end{itemize}

\begin{itemize}
  \item 1 o 100%: Seguro (siempre, siempre pasa).
\end{itemize}

\begin{alertblock}{💡 Nota importante}
¡Toda probabilidad vive en esta escala! 
\end{alertblock}

\end{frame}

\begin{frame}[fragile]{Visualización - La Escalera de la Probabilidad}
\begin{itemize}
  \item Imposible: 0% (Que un pez vuele )
\end{itemize}

\begin{itemize}
  \item Poco probable: 25% (Sacar un 6 en un dado )
\end{itemize}

\begin{itemize}
  \item Igual de probable: 50% (Que salga 'cara' en una moneda )
\end{itemize}

\begin{itemize}
  \item Muy probable: 75% (Que mañana salga el sol )
\end{itemize}

\begin{itemize}
  \item Seguro: 100% (Que después del Lunes venga el Martes)
\end{itemize}

\end{frame}

\begin{frame}[fragile]{Palabras Clave del Juego}
\begin{itemize}
  \item  Experimento: La acción que hacemos (lanzar un dado).
\end{itemize}

\begin{itemize}
  \item  Resultado: Lo que puede salir (los números 1, 2, 3, 4, 5, 6).
\end{itemize}

\begin{itemize}
  \item  Espacio Muestral: ¡TODOS los resultados posibles! {1,2,3,4,5,6}
\end{itemize}

\begin{itemize}
  \item  Evento: El resultado que nos interesa (que salga un número par {2,4,6}).
\end{itemize}

\begin{alertblock}{💡 Nota importante}
¡Con estas palabras, hablamos el idioma de las probabilidades!
\end{alertblock}

\end{frame}

\begin{frame}[fragile]{La Fórmula Mágica de la Probabilidad}
\begin{center}
\colorbox{colorFondo}{%
  \parbox{0.8\textwidth}{%
    \begin{center}
    \Large\color{colorPrimario}
    $Probabilidad = (Cosas que busco) / (Total de posibilidades)$
    \end{center}
  }%
}
\end{center}

\begin{itemize}
  \item Cosas que busco: los resultados que nos interesan (casos favorables).
\end{itemize}

\begin{itemize}
  \item Total de posibilidades: todos los resultados que pueden ocurrir (casos totales).
\end{itemize}

\begin{alertblock}{💡 Nota importante}
¡Es como contar tus chances de ganar! 
\end{alertblock}

\end{frame}

\begin{frame}[fragile]{Ejemplo con un Dado }
\begin{exampleblock}{🎯 Problema}
Lanzamos un dado. ¿Cuál es la probabilidad de sacar un número par?
\end{exampleblock}

\begin{itemize}
  \item Total de posibilidades: 6 (puede salir 1, 2, 3, 4, 5 o 6).
\end{itemize}

\begin{itemize}
  \item Cosas que busco (pares): 3 (el 2, 4 y 6).
\end{itemize}

\begin{center}
\colorbox{colorFondo}{%
  \parbox{0.9\textwidth}{%
    \begin{align*}
    Probabilidad = 3 / 6 = 1/2 = 50%
    \end{align*}
  }%
}
\end{center}

\begin{alertblock}{💡 Nota importante}
¡Tienes un 50% de probabilidad! ¡La mitad de las veces ganarás!
\end{alertblock}

\end{frame}

\begin{frame}[fragile]{Visualización - Pares vs. Impares}
\begin{itemize}
  \item En un dado, la mitad de los números son pares y la otra mitad impares:
\end{itemize}

\begin{itemize}
  \item • Pares: 2, 4, 6 (3 números = 50% de chance)
\end{itemize}

\begin{itemize}
  \item • Impares: 1, 3, 5 (3 números = 50% de chance)
\end{itemize}

\begin{alertblock}{💡 Nota importante}
¡Por eso es una apuesta 50/50!
\end{alertblock}

\end{frame}

\begin{frame}[fragile]{Ejemplo con Cartas }
\begin{exampleblock}{🎯 Problema}
De una baraja española (40 cartas), ¿cuál es la probabilidad de sacar un As?
\end{exampleblock}

\begin{itemize}
  \item Total de posibilidades: 40 cartas.
\end{itemize}

\begin{itemize}
  \item Cosas que busco (Ases): Hay 4 Ases en la baraja.
\end{itemize}

\begin{center}
\colorbox{colorFondo}{%
  \parbox{0.9\textwidth}{%
    \begin{align*}
    Probabilidad de As = 4 / 40 = 1/10 = 10%
    \end{align*}
  }%
}
\end{center}

\begin{itemize}
  \item ¡Tienes un 10% de probabilidad de sacar un As!
\end{itemize}

\end{frame}

\begin{frame}[fragile]{Regla del Complemento: ¿Y si NO pasa?}
\begin{itemize}
  \item El complemento es todo lo contrario a lo que buscas.
\end{itemize}

\begin{center}
\colorbox{colorFondo}{%
  \parbox{0.8\textwidth}{%
    \begin{center}
    \Large\color{colorPrimario}
    $P(que NO pase) = 1 - P(que SÍ pase)$
    \end{center}
  }%
}
\end{center}

\begin{alertblock}{💡 Nota importante}
¡Si hay 30% de chance de lluvia, hay 70% de que NO llueva!
\end{alertblock}

\begin{exampleblock}{✨ Ejemplo}
Si P(ganar) = 0.2, entonces P(no ganar) = 1 - 0.2 = 0.8
\end{exampleblock}

\end{frame}

\begin{frame}[fragile]{Visualización - El Complemento}
\begin{itemize}
  \item Si hay 30% de probabilidad de que llueva:
\end{itemize}

\begin{itemize}
  \item • Lluvia: 30%
\end{itemize}

\begin{itemize}
  \item • No Lluvia: 70% (el resto)
\end{itemize}

\begin{itemize}
  \item Total: 30% + 70% = 100%
\end{itemize}

\begin{alertblock}{💡 Nota importante}
¡Lo que pasa y lo que no pasa siempre suman el 100%!
\end{alertblock}

\end{frame}

\begin{frame}[fragile]{Ejemplo del Complemento}
\begin{exampleblock}{🎯 Problema}
En un dado, ¿cuál es la probabilidad de NO sacar un 6?
\end{exampleblock}

\begin{itemize}
  \item Forma larga: Contar los que no son 6 (1,2,3,4,5) → 5/6.
\end{itemize}

\begin{itemize}
  \item Forma rápida (con complemento):
\end{itemize}

\begin{itemize}
  \item P(sacar 6) = 1/6
\end{itemize}

\begin{center}
\colorbox{colorFondo}{%
  \parbox{0.9\textwidth}{%
    \begin{align*}
    P(NO sacar 6) = 1 - 1/6 = 5/6
    \end{align*}
  }%
}
\end{center}

\begin{alertblock}{💡 Nota importante}
¡A veces es más fácil calcular lo contrario y restar! 
\end{alertblock}

\end{frame}

\begin{frame}[fragile]{Regla de la Suma: ¿Probabilidad de A o B?}
\begin{itemize}
  \item Se usa para calcular la probabilidad de que ocurra el evento A O el evento B.
\end{itemize}

\begin{alertblock}{💡 Nota importante}
¡OJO! Si los eventos se pueden cruzar, hay que tener cuidado.
\end{alertblock}

\begin{center}
\colorbox{colorFondo}{%
  \parbox{0.8\textwidth}{%
    \begin{center}
    \Large\color{colorPrimario}
    $P(A o B) = P(A) + P(B) - P(A y B juntos)$
    \end{center}
  }%
}
\end{center}

\begin{itemize}
  \item ¿Por qué restamos? ¡Para no contar dos veces lo que está en ambos grupos!
\end{itemize}

\end{frame}

\begin{frame}[fragile]{Eventos que NO se pueden cruzar}
\begin{itemize}
  \item Se llaman 'mutuamente excluyentes'. ¡No pueden pasar a la vez!
\end{itemize}

\begin{exampleblock}{✨ Ejemplo}
Al lanzar un dado, no puedes sacar un 2 y un 5 al mismo tiempo.
\end{exampleblock}

\begin{itemize}
  \item Para estos casos, la fórmula es más fácil:
\end{itemize}

\begin{center}
\colorbox{colorFondo}{%
  \parbox{0.8\textwidth}{%
    \begin{center}
    \Large\color{colorPrimario}
    $P(A o B) = P(A) + P(B)$
    \end{center}
  }%
}
\end{center}

\begin{alertblock}{💡 Nota importante}
¡Solo sumamos porque no hay nada que contar dos veces! 
\end{alertblock}

\end{frame}

\begin{frame}[fragile]{Ejemplo de Regla de la Suma (Simple)}
\begin{exampleblock}{🎯 Problema}
En una baraja (40 cartas), ¿probabilidad de sacar un As O un Rey?
\end{exampleblock}

\begin{itemize}
  \item Son eventos que no se cruzan (una carta no es As y Rey a la vez).
\end{itemize}

\begin{itemize}
  \item P(As) = 4/40
\end{itemize}

\begin{itemize}
  \item P(Rey) = 4/40
\end{itemize}

\begin{center}
\colorbox{colorFondo}{%
  \parbox{0.9\textwidth}{%
    \begin{align*}
    P(As o Rey) = 4/40 + 4/40 = 8/40 = 20%
    \end{align*}
  }%
}
\end{center}

\end{frame}

\begin{frame}[fragile]{Ejemplo de Regla de la Suma (con cruce)}
\begin{exampleblock}{🎯 Problema}
En una baraja (52 cartas), ¿probabilidad de sacar un Corazón  O una Figura (J,Q,K)?
\end{exampleblock}

\begin{itemize}
  \item P(Corazón) = 13/52
\end{itemize}

\begin{itemize}
  \item P(Figura) = 12/52
\end{itemize}

\begin{itemize}
  \item ¡PERO! Hay 3 cartas que son Corazón Y Figura (J, Q, K). Las contamos dos veces.
\end{itemize}

\begin{itemize}
  \item P(Corazón y Figura) = 3/52
\end{itemize}

\begin{center}
\colorbox{colorFondo}{%
  \parbox{0.9\textwidth}{%
    \begin{align*}
    P( o Figura) = (13/52) + (12/52) - (3/52) = 22/52
    \end{align*}
  }%
}
\end{center}

\end{frame}

\begin{frame}[fragile]{Regla de la Multiplicación: ¿A y B?}
\begin{itemize}
  \item Se usa para calcular la probabilidad de que ocurra A Y LUEGO ocurra B.
\end{itemize}

\begin{alertblock}{💡 Nota importante}
¡Piensa en eventos que pasan en cadena! 
\end{alertblock}

\begin{itemize}
  \item La idea clave es: ¿el primer evento afecta al segundo?
\end{itemize}

\end{frame}

\begin{frame}[fragile]{Eventos Independientes (no se afectan)}
\begin{itemize}
  \item El resultado del primer evento NO cambia las probabilidades del segundo.
\end{itemize}

\begin{exampleblock}{✨ Ejemplo}
Lanzar una moneda dos veces. El primer resultado no afecta al segundo.
\end{exampleblock}

\begin{itemize}
  \item Si son independientes, la fórmula es fácil:
\end{itemize}

\begin{center}
\colorbox{colorFondo}{%
  \parbox{0.8\textwidth}{%
    \begin{center}
    \Large\color{colorPrimario}
    $P(A y B) = P(A) x P(B)$
    \end{center}
  }%
}
\end{center}

\begin{alertblock}{💡 Nota importante}
¡Solo multiplicas sus probabilidades! 
\end{alertblock}

\end{frame}

\begin{frame}[fragile]{Ejemplo de Eventos Independientes}
\begin{exampleblock}{🎯 Problema}
Lanzamos dos monedas. ¿Probabilidad de que ambas salgan 'cara'?
\end{exampleblock}

\end{frame}

\begin{frame}[fragile]{Eventos Dependientes (sí se afectan)}
\begin{itemize}
  \item El resultado del primer evento SÍ cambia las probabilidades del segundo.
\end{itemize}

\begin{exampleblock}{✨ Ejemplo}
Sacar dos cartas de una baraja SIN devolver la primera.
\end{exampleblock}

\begin{exampleblock}{🎯 Problema}
Urna con 5 bolas rojas y 3 azules. Sacas 2 SIN devolverlas.
\end{exampleblock}

\begin{itemize}
  \item ¿Probabilidad de sacar 2 rojas seguidas?
\end{itemize}

\begin{itemize}
  \item 1ra bola: P(Roja) = 5/8
\end{itemize}

\begin{itemize}
  \item 2da bola: Como ya sacaste una roja, quedan 4 rojas y 7 bolas en total.
\end{itemize}

\begin{itemize}
  \item P(2da Roja) = 4/7
\end{itemize}

\begin{center}
\colorbox{colorFondo}{%
  \parbox{0.9\textwidth}{%
    \begin{align*}
    P(Roja y Roja) = (5/8) x (4/7) = 20/56 ≈ 36%
    \end{align*}
  }%
}
\end{center}

\end{frame}

\begin{frame}[fragile]{Con Devolución vs. Sin Devolución}
\begin{itemize}
  \item CON devolución (o reposición): Los eventos son INDEPENDIENTES. Las probabilidades NO cambian.
\end{itemize}

\begin{itemize}
  \item SIN devolución (o reposición): Los eventos son DEPENDIENTES. Las probabilidades SÍ cambian.
\end{itemize}

\begin{alertblock}{💡 Nota importante}
¡Fíjate siempre si las cosas se devuelven o no! Es la clave. 
\end{alertblock}

\end{frame}

\begin{frame}[fragile]{Problema de la Moneda }
\begin{exampleblock}{🎯 Problema}
Lanzas una moneda 3 veces. ¿Probabilidad de sacar 3 caras seguidas?
\end{exampleblock}

\begin{itemize}
  \item Cada lanzamiento es independiente del anterior.
\end{itemize}

\begin{itemize}
  \item P(cara) en cada uno es 1/2.
\end{itemize}

\begin{center}
\colorbox{colorFondo}{%
  \parbox{0.9\textwidth}{%
    \begin{align*}
    P(Cara, Cara, Cara) = 1/2 x 1/2 x 1/2 = 1/8
    \end{align*}
  }%
}
\end{center}

\begin{alertblock}{💡 Nota importante}
¡Solo un 12.5% de probabilidad! No es tan fácil como parece. 
\end{alertblock}

\end{frame}

\begin{frame}[fragile]{Problema Combinado: ¡Aplica todo!}
\begin{exampleblock}{🎯 Problema}
En una bolsa hay 5 canicas rojas y 3 azules.
\end{exampleblock}

\begin{itemize}
  \item a) ¿Probabilidad de sacar una roja?
\end{itemize}

\begin{itemize}
  \item b) ¿Probabilidad de sacar una roja O una azul?
\end{itemize}

\begin{itemize}
  \item c) ¿Probabilidad de sacar 2 rojas seguidas SIN devolverlas?
\end{itemize}

\begin{alertblock}{💡 Nota importante}
¡Vamos a resolverlo paso a paso! 
\end{alertblock}

\end{frame}

\begin{frame}[fragile]{Solución Parte a) y b)}
\begin{itemize}
  \item Total de canicas: 5 + 3 = 8
\end{itemize}

\begin{itemize}
  \item a) P(Roja) = (canicas rojas) / (total) = 5/8
\end{itemize}

\begin{itemize}
  \item b) Roja O Azul... ¡son las únicas que hay! Es 100% seguro que sacarás una de esas.
\end{itemize}

\begin{itemize}
  \item Usando la fórmula (no se cruzan):
\end{itemize}

\begin{center}
\colorbox{colorFondo}{%
  \parbox{0.9\textwidth}{%
    \begin{align*}
    P(Roja o Azul) = P(Roja) + P(Azul) = 5/8 + 3/8 = 8/8 = 1
    \end{align*}
  }%
}
\end{center}

\begin{alertblock}{💡 Nota importante}
¡Una probabilidad de 1 significa que es un evento seguro!
\end{alertblock}

\end{frame}

\begin{frame}[fragile]{Visualización - Distribución de Canicas}
\begin{itemize}
  \item Distribución en la bolsa:
\end{itemize}

\begin{itemize}
  \item • Rojas: 5 de 8 = 62.5% de la bolsa.
\end{itemize}

\begin{itemize}
  \item • Azules: 3 de 8 = 37.5% de la bolsa.
\end{itemize}

\begin{itemize}
  \item Total: 62.5% + 37.5% = 100%
\end{itemize}

\end{frame}

\begin{frame}[fragile]{Solución Parte c)}
\begin{itemize}
  \item Son eventos dependientes (SIN devolución).
\end{itemize}

\begin{itemize}
  \item P(1ra sea Roja) = 5/8
\end{itemize}

\begin{itemize}
  \item Después de sacar una roja, quedan 4 rojas y 7 canicas en total.
\end{itemize}

\begin{itemize}
  \item P(2da sea Roja) = 4/7
\end{itemize}

\begin{center}
\colorbox{colorFondo}{%
  \parbox{0.9\textwidth}{%
    \begin{align*}
    c) P(Roja y luego Roja) = (5/8) x (4/7) = 20/56 ≈ 35.7%
    \end{align*}
  }%
}
\end{center}

\begin{alertblock}{💡 Nota importante}
¡La probabilidad bajó en el segundo intento! 
\end{alertblock}

\end{frame}

\begin{frame}[fragile]{Tips para Resolver Problemas}
\begin{itemize}
  \item 1. Lee con calma: ¿El problema dice 'O' o dice 'Y'?
\end{itemize}

\begin{itemize}
  \item 2. ¿Los eventos se afectan entre sí? (¿Es con o sin devolución?)
\end{itemize}

\begin{itemize}
  \item 3. Escribe las probabilidades de cada evento simple primero.
\end{itemize}

\begin{itemize}
  \item 4. Decide qué regla usar: Suma (para 'O') o Multiplicación (para 'Y').
\end{itemize}

\begin{itemize}
  \item 5. ¡Simplifica el resultado si puedes!
\end{itemize}

\end{frame}

\begin{frame}[fragile]{Ejercicio Final de Práctica}
\begin{exampleblock}{🎯 Problema}
En una caja hay 4 galletas de chocolate, 3 de vainilla y 2 de fresa. Sacas una sin mirar.
\end{exampleblock}

\begin{itemize}
  \item a) ¿Probabilidad de que sea de vainilla?
\end{itemize}

\begin{itemize}
  \item b) ¿Probabilidad de que sea de chocolate O de fresa?
\end{itemize}

\begin{itemize}
  \item c) Si te comes la primera (es de vainilla), ¿cuál es la probabilidad de que la segunda que saques sea de chocolate?
\end{itemize}

\end{frame}

\begin{frame}[fragile]{Solución del Ejercicio Final}
\begin{itemize}
  \item Total de galletas: 4 + 3 + 2 = 9
\end{itemize}

\begin{itemize}
  \item a) P(Vainilla) = 3/9 = 1/3
\end{itemize}

\begin{itemize}
  \item b) P(Choco O Fresa) = P(Choco) + P(Fresa) = 4/9 + 2/9 = 6/9 = 2/3
\end{itemize}

\begin{itemize}
  \item c) Después de comer una de vainilla, quedan 8 galletas (4 de choco).
\end{itemize}

\begin{itemize}
  \item P(2da sea Choco) = 4/8 = 1/2
\end{itemize}

\end{frame}

\begin{frame}[fragile]{¡Resumen de la Misión!}
\begin{itemize}
  \item  Probabilidad Básica: (lo que busco) / (el total).
\end{itemize}

\begin{itemize}
  \item  Regla del Complemento: Para calcular la chance de que algo NO pase.
\end{itemize}

\begin{itemize}
  \item  Regla de la Suma (O): Sumamos probabilidades (¡cuidado con los cruces!).
\end{itemize}

\begin{itemize}
  \item  Regla de la Multiplicación (Y): Multiplicamos probabilidades (¡cuidado si se afectan!).
\end{itemize}

\begin{alertblock}{💡 Nota importante}
¡Ahora tienes los superpoderes para calcular probabilidades! 
\end{alertblock}

\end{frame}


% Diapositiva final atractiva
\begin{frame}[plain]
\begin{tikzpicture}[remember picture,overlay]
  \fill[colorAccento!20] (current page.south west) rectangle (current page.north east);
  \node[font=\Huge\bfseries,text=colorPrimario] at (current page.center) {¿Preguntas? 🤔};
  \node[font=\large,text=colorSecundario,below=1.5cm] at (current page.center) {¡Sigue aprendiendo! 🚀};
\end{tikzpicture}
\end{frame}

\end{document}