\documentclass[aspectratio=169]{beamer}

% Tema moderno y lúdico
\usetheme{Boadilla}
\usecolortheme{dolphin}

% Paquetes esenciales
\usepackage[utf8]{inputenc}
\usepackage[spanish]{babel}
\usepackage{amsmath}
\usepackage{amssymb}
\usepackage{tikz}
\usetikzlibrary{shapes, arrows, positioning, calc}
\usepackage{pgfplots}
\pgfplotsset{compat=1.18}
\usepackage{pgf-pie}
\usepackage{booktabs}
\usepackage{xcolor}
\usepackage{colortbl}
\usepackage{newunicodechar}

% Configurar emojis como texto simple
\newunicodechar{🎲}{\textsf{[dado]}}
\newunicodechar{🎯}{\textsf{[objetivo]}}
\newunicodechar{📋}{\textsf{[lista]}}
\newunicodechar{⭐}{\textsf{[estrella]}}
\newunicodechar{🔮}{\textsf{[bola]}}
\newunicodechar{📏}{\textsf{[regla]}}
\newunicodechar{🃏}{\textsf{[carta]}}
\newunicodechar{🔄}{\textsf{[ciclo]}}
\newunicodechar{🔢}{\textsf{[numeros]}}
\newunicodechar{➕}{\textsf{[+]}}
\newunicodechar{✖}{\textsf{[x]}}
\newunicodechar{🔗}{\textsf{[cadena]}}
\newunicodechar{🪙}{\textsf{[moneda]}}
\newunicodechar{🔑}{\textsf{[llave]}}
\newunicodechar{🔍}{\textsf{[lupa]}}
\newunicodechar{📊}{\textsf{[grafico]}}
\newunicodechar{🎴}{\textsf{[cartas]}}
\newunicodechar{💪}{\textsf{[fuerza]}}
\newunicodechar{🌟}{\textsf{[estrella]}}
\newunicodechar{💡}{\textsf{[idea]}}
\newunicodechar{✨}{\textsf{[brillos]}}
\newunicodechar{✅}{\textsf{[check]}}
\newunicodechar{🎓}{\textsf{[gorro]}}
\newunicodechar{🤔}{\textsf{[pensar]}}
\newunicodechar{🚀}{\textsf{[cohete]}}
\newunicodechar{♥}{\ensuremath{\heartsuit}}
\newunicodechar{₁}{\textsubscript{1}}
\newunicodechar{₂}{\textsubscript{2}}
\newunicodechar{✖}{\textsf{x}}
\newunicodechar{✓}{\textsf{[ok]}}
\newunicodechar{⚫}{\textsf{[punto]}}
\newunicodechar{👥}{\textsf{[grupo]}}
\newunicodechar{️}{}

% Colores personalizados vibrantes y lúdicos
\definecolor{colorPrimario}{RGB}{41, 128, 185}      % Azul vibrante
\definecolor{colorSecundario}{RGB}{231, 76, 60}     % Rojo coral
\definecolor{colorAccento}{RGB}{46, 204, 113}       % Verde esmeralda
\definecolor{colorAdvertencia}{RGB}{241, 196, 15}   % Amarillo dorado
\definecolor{colorMorado}{RGB}{155, 89, 182}        % Morado amigable
\definecolor{colorNaranja}{RGB}{230, 126, 34}       % Naranja cálido
\definecolor{colorFondo}{RGB}{236, 240, 241}        % Gris claro de fondo

% Configuración del tema
\setbeamercolor{structure}{fg=colorPrimario}
\setbeamercolor{palette primary}{bg=colorPrimario,fg=white}
\setbeamercolor{palette secondary}{bg=colorSecundario,fg=white}
\setbeamercolor{palette tertiary}{bg=colorAccento,fg=white}
\setbeamercolor{block title}{bg=colorPrimario!90,fg=white}
\setbeamercolor{block body}{bg=colorPrimario!10,fg=black}
\setbeamercolor{block title example}{bg=colorAccento!90,fg=white}
\setbeamercolor{block body example}{bg=colorAccento!10,fg=black}
\setbeamercolor{block title alerted}{bg=colorSecundario!90,fg=white}
\setbeamercolor{block body alerted}{bg=colorSecundario!10,fg=black}

% Plantillas personalizadas
\setbeamertemplate{navigation symbols}{}
\setbeamertemplate{footline}[frame number]
\setbeamertemplate{itemize items}[circle]
\setbeamertemplate{enumerate items}[circle]
\setbeamerfont{title}{size=\huge,series=\bfseries}
\setbeamerfont{frametitle}{size=\Large,series=\bfseries}

% Comandos personalizados para énfasis
\renewcommand{\alert}[1]{\textcolor{colorSecundario}{\textbf{#1}}}
\newcommand{\highlight}[1]{\textcolor{colorPrimario}{\textbf{#1}}}
\newcommand{\importante}[1]{\textcolor{colorAdvertencia}{\textbf{#1}}}
\newcommand{\exito}[1]{\textcolor{colorAccento}{\textbf{#1}}}

% Variables del documento (serán reemplazadas por Jinja2)
\title{Introducción a la Probabilidad y Estadística}
\subtitle{¡Descubriendo el mundo de los números y las posibilidades!}
\author{Probabilidad y Estadística}
\institute{Aprendiendo con diversión 🎓}
\date{\today}

\begin{document}

% Página de título con diseño atractivo
\begin{frame}[plain]
\begin{tikzpicture}[remember picture,overlay]
  % Fondo decorativo
  \fill[colorPrimario!20] (current page.south west) rectangle (current page.north east);
  \fill[colorAccento!30] (current page.south west) -- (current page.south east) -- 
        ([yshift=-3cm]current page.north east) -- ([yshift=-3cm]current page.north west) -- cycle;
\end{tikzpicture}
\titlepage
\end{frame}

% Diapositivas generadas dinámicamente
\begin{frame}[fragile]{¿Qué vamos a aprender?}
\begin{alertblock}{💡 Nota importante}
¡Bienvenidos a una aventura numérica! 🎲📊
\end{alertblock}

\begin{itemize}
  \item Dos mundos fascinantes:
\end{itemize}

\begin{itemize}
  \item 📊 Estadística: entender lo que ya pasó
\end{itemize}

\begin{itemize}
  \item 🎲 Probabilidad: predecir lo que puede pasar
\end{itemize}

\begin{exampleblock}{✨ Ejemplo}
Como un detective que analiza pistas (estadística) y adivina quién es el culpable (probabilidad)
\end{exampleblock}

\end{frame}

\begin{frame}[fragile]{¿Qué es la Estadística?}
\begin{itemize}
  \item Es el arte de recolectar, organizar y analizar datos
\end{itemize}

\begin{itemize}
  \item Nos ayuda a entender el pasado y el presente
\end{itemize}

\begin{alertblock}{💡 Nota importante}
¡Es como ser un detective de números! 🔍
\end{alertblock}

\begin{exampleblock}{✨ Ejemplo}
Contar cuántos estudiantes prefieren pizza vs hamburguesa
\end{exampleblock}

\end{frame}

\begin{frame}[fragile]{La Estadística en tu Vida Diaria}
\begin{itemize}
  \item 📱 Redes sociales: conteo de likes y seguidores
\end{itemize}

\begin{itemize}
  \item ⚽ Deportes: estadísticas de goles y victorias
\end{itemize}

\begin{itemize}
  \item 🎮 Videojuegos: puntuaciones y rankings
\end{itemize}

\begin{itemize}
  \item 📺 Netflix: series más vistas
\end{itemize}

\begin{alertblock}{💡 Nota importante}
¡Estás rodeado de estadísticas sin darte cuenta!
\end{alertblock}

\end{frame}

\begin{frame}[fragile]{Ejemplo Divertido de Estadística}
\begin{exampleblock}{🎯 Problema}
Preguntamos a 10 amigos su sabor de helado favorito
\end{exampleblock}

\begin{center}
\begin{tabular}{cc}
\toprule
\rowcolor{colorPrimario!20}
\textbf{Sabor} & \textbf{Votos} \\
\midrule
\rowcolor{colorFondo}
Chocolate & 4 \\
Vainilla & 3 \\
\rowcolor{colorFondo}
Fresa & 2 \\
Limón & 1 \\
\bottomrule
\end{tabular}
\end{center}

\begin{center}
\colorbox{colorFondo}{%
  \parbox{0.9\textwidth}{%
    \begin{align*}
    Conclusión: ¡El chocolate es el campeón! 🍫
    \end{align*}
  }%
}
\end{center}

\end{frame}

\begin{frame}[fragile]{¿Qué es la Probabilidad?}
\begin{itemize}
  \item Es calcular qué tan posible es que algo ocurra
\end{itemize}

\begin{itemize}
  \item Trabaja con eventos FUTUROS o inciertos
\end{itemize}

\begin{alertblock}{💡 Nota importante}
¡Es como predecir el futuro con matemáticas! 🔮
\end{alertblock}

\begin{exampleblock}{✨ Ejemplo}
¿Qué probabilidad hay de que llueva mañana?
\end{exampleblock}

\end{frame}

\begin{frame}[fragile]{La Probabilidad en tu Vida}
\begin{itemize}
  \item 🎲 Juegos de mesa: lanzar dados
\end{itemize}

\begin{itemize}
  \item 🃏 Cartas: ¿me saldrá un as?
\end{itemize}

\begin{itemize}
  \item ⚽ Deportes: ¿ganará mi equipo?
\end{itemize}

\begin{itemize}
  \item 🌦️ Clima: pronóstico del tiempo
\end{itemize}

\begin{itemize}
  \item 🎰 Sorteos y rifas
\end{itemize}

\begin{alertblock}{💡 Nota importante}
¡La probabilidad está en cada decisión!
\end{alertblock}

\end{frame}

\begin{frame}[fragile]{Ejemplo Divertido de Probabilidad}
\begin{exampleblock}{🎯 Problema}
Tienes una bolsa con 10 caramelos: 6 rojos y 4 azules
\end{exampleblock}

\begin{itemize}
  \item Sin mirar, sacas uno. ¿Qué color es más probable?
\end{itemize}

\begin{center}
\colorbox{colorFondo}{%
  \parbox{0.9\textwidth}{%
    \begin{align*}
    Rojo: 6 de cada 10 (60%)
    \end{align*}
  }%
}
\end{center}

\begin{center}
\colorbox{colorFondo}{%
  \parbox{0.9\textwidth}{%
    \begin{align*}
    Azul: 4 de cada 10 (40%)
    \end{align*}
  }%
}
\end{center}

\begin{itemize}
  \item ¡Es más probable sacar uno rojo! 🔴
\end{itemize}

\end{frame}

\begin{frame}[fragile]{Diferencias Clave}
\begin{center}
\begin{tabular}{cc}
\toprule
\rowcolor{colorPrimario!20}
\textbf{Estadística} & \textbf{Probabilidad} \\
\midrule
\rowcolor{colorFondo}
Analiza datos del PASADO & Predice el FUTURO \\
¿Qué pasó? & ¿Qué puede pasar? \\
\rowcolor{colorFondo}
Certeza & Incertidumbre \\
Temperaturas del mes & Pronóstico del tiempo \\
\bottomrule
\end{tabular}
\end{center}

\begin{alertblock}{💡 Nota importante}
¡Son dos caras de la misma moneda! 🪙
\end{alertblock}

\end{frame}

\begin{frame}[fragile]{¿Por qué son Importantes?}
\begin{itemize}
  \item 🏥 Medicina: estudios de efectividad de tratamientos
\end{itemize}

\begin{itemize}
  \item 💼 Negocios: decidir qué productos vender
\end{itemize}

\begin{itemize}
  \item 🎓 Educación: mejorar métodos de enseñanza
\end{itemize}

\begin{itemize}
  \item 🌍 Ciencia: descubrir patrones en la naturaleza
\end{itemize}

\begin{alertblock}{💡 Nota importante}
¡Ayudan a tomar mejores decisiones!
\end{alertblock}

\end{frame}

\begin{frame}[fragile]{Vocabulario Básico}
\begin{itemize}
  \item 📊 Datos: información que recolectamos
\end{itemize}

\begin{itemize}
  \item 👥 Población: todos los elementos que estudiamos
\end{itemize}

\begin{itemize}
  \item 🎯 Muestra: parte de la población
\end{itemize}

\begin{itemize}
  \item 🎲 Evento: algo que puede ocurrir
\end{itemize}

\begin{itemize}
  \item 🎯 Experimento: acción que produce resultados
\end{itemize}

\begin{exampleblock}{✨ Ejemplo}
Experimento: lanzar una moneda. Evento: sale cara
\end{exampleblock}

\end{frame}

\begin{frame}[fragile]{Un Experimento Divertido}
\begin{exampleblock}{🎯 Problema}
Lanza una moneda 10 veces y anota los resultados
\end{exampleblock}

\begin{itemize}
  \item Estadística: cuenta cuántas caras y cuántas sellos
\end{itemize}

\begin{itemize}
  \item Probabilidad: ¿cuál es la chance teórica?
\end{itemize}

\begin{center}
\colorbox{colorFondo}{%
  \parbox{0.9\textwidth}{%
    \begin{align*}
    Teóricamente: 50% cara, 50% sello
    \end{align*}
  }%
}
\end{center}

\begin{alertblock}{💡 Nota importante}
¿Coincide tu experimento con la teoría?
\end{alertblock}

\end{frame}

\begin{frame}[fragile]{Escalera de la Probabilidad}
\begin{itemize}
  \item Imposible (0%): que salga un 7 en un dado normal 🎲
\end{itemize}

\begin{itemize}
  \item Poco probable (25%): sacar una carta de corazones ♥️
\end{itemize}

\begin{itemize}
  \item Igual de probable (50%): que salga cara 🪙
\end{itemize}

\begin{itemize}
  \item Muy probable (75%): NO sacar un 6 en un dado
\end{itemize}

\begin{itemize}
  \item Seguro (100%): que el sol salga mañana ☀️
\end{itemize}

\begin{alertblock}{💡 Nota importante}
¡Todo evento tiene su lugar en la escalera!
\end{alertblock}

\end{frame}

\begin{frame}[fragile]{Actividad Práctica}
\begin{exampleblock}{🎯 Problema}
Piensa en estas situaciones y clasifícalas:
\end{exampleblock}

\begin{itemize}
  \item 1. ¿Es estadística o probabilidad?
\end{itemize}

\begin{itemize}
  \item 2. Calificaciones del año pasado
\end{itemize}

\begin{itemize}
  \item 3. Probabilidad de ganar un juego
\end{itemize}

\begin{itemize}
  \item 4. Número de goles en el mundial anterior
\end{itemize}

\begin{itemize}
  \item 5. Chance de que nieve en tu ciudad
\end{itemize}

\end{frame}

\begin{frame}[fragile]{¿Listo para la Aventura?}
\begin{itemize}
  \item ✅ Ya sabes qué es la estadística
\end{itemize}

\begin{itemize}
  \item ✅ Ya sabes qué es la probabilidad
\end{itemize}

\begin{itemize}
  \item ✅ Viste ejemplos de la vida real
\end{itemize}

\begin{alertblock}{💡 Nota importante}
¡Ahora estás listo para profundizar en cada tema!
\end{alertblock}

\begin{itemize}
  \item En las próximas clases aprenderemos:
\end{itemize}

\begin{itemize}
  \item 📊 Tablas y gráficos
\end{itemize}

\begin{itemize}
  \item 📏 Medidas estadísticas
\end{itemize}

\begin{itemize}
  \item 🎲 Cálculos de probabilidad
\end{itemize}

\end{frame}


% Diapositiva final atractiva
\begin{frame}[plain]
\begin{tikzpicture}[remember picture,overlay]
  \fill[colorAccento!20] (current page.south west) rectangle (current page.north east);
  \node[font=\Huge\bfseries,text=colorPrimario] at (current page.center) {¿Preguntas? 🤔};
  \node[font=\large,text=colorSecundario,below=1.5cm] at (current page.center) {¡Sigue aprendiendo! 🚀};
\end{tikzpicture}
\end{frame}

\end{document}