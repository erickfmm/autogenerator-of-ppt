\documentclass[aspectratio=169]{beamer}

% Tema moderno y lúdico
\usetheme{Boadilla}
\usecolortheme{dolphin}

% Paquetes esenciales
\usepackage[utf8]{inputenc}
\usepackage[spanish]{babel}
\usepackage{amsmath}
\usepackage{amssymb}
\usepackage{tikz}
\usetikzlibrary{shapes, arrows, positioning, calc}
\usepackage{pgfplots}
\pgfplotsset{compat=1.18}
\usepackage{pgf-pie}
\usepackage{booktabs}
\usepackage{xcolor}
\usepackage{colortbl}
\usepackage{newunicodechar}

% Configurar emojis como texto simple
\newunicodechar{🎲}{\textsf{[dado]}}
\newunicodechar{🎯}{\textsf{[objetivo]}}
\newunicodechar{📋}{\textsf{[lista]}}
\newunicodechar{⭐}{\textsf{[estrella]}}
\newunicodechar{🔮}{\textsf{[bola]}}
\newunicodechar{📏}{\textsf{[regla]}}
\newunicodechar{🃏}{\textsf{[carta]}}
\newunicodechar{🔄}{\textsf{[ciclo]}}
\newunicodechar{🔢}{\textsf{[numeros]}}
\newunicodechar{➕}{\textsf{[+]}}
\newunicodechar{✖}{\textsf{[x]}}
\newunicodechar{🔗}{\textsf{[cadena]}}
\newunicodechar{🪙}{\textsf{[moneda]}}
\newunicodechar{🔑}{\textsf{[llave]}}
\newunicodechar{🔍}{\textsf{[lupa]}}
\newunicodechar{📊}{\textsf{[grafico]}}
\newunicodechar{🎴}{\textsf{[cartas]}}
\newunicodechar{💪}{\textsf{[fuerza]}}
\newunicodechar{🌟}{\textsf{[estrella]}}
\newunicodechar{💡}{\textsf{[idea]}}
\newunicodechar{✨}{\textsf{[brillos]}}
\newunicodechar{✅}{\textsf{[check]}}
\newunicodechar{🎓}{\textsf{[gorro]}}
\newunicodechar{🤔}{\textsf{[pensar]}}
\newunicodechar{🚀}{\textsf{[cohete]}}
\newunicodechar{♥}{\ensuremath{\heartsuit}}
\newunicodechar{₁}{\textsubscript{1}}
\newunicodechar{₂}{\textsubscript{2}}
\newunicodechar{✖}{\textsf{x}}
\newunicodechar{✓}{\textsf{[ok]}}
\newunicodechar{⚫}{\textsf{[punto]}}
\newunicodechar{👥}{\textsf{[grupo]}}
\newunicodechar{️}{}

% Colores personalizados vibrantes y lúdicos
\definecolor{colorPrimario}{RGB}{41, 128, 185}      % Azul vibrante
\definecolor{colorSecundario}{RGB}{231, 76, 60}     % Rojo coral
\definecolor{colorAccento}{RGB}{46, 204, 113}       % Verde esmeralda
\definecolor{colorAdvertencia}{RGB}{241, 196, 15}   % Amarillo dorado
\definecolor{colorMorado}{RGB}{155, 89, 182}        % Morado amigable
\definecolor{colorNaranja}{RGB}{230, 126, 34}       % Naranja cálido
\definecolor{colorFondo}{RGB}{236, 240, 241}        % Gris claro de fondo

% Configuración del tema
\setbeamercolor{structure}{fg=colorPrimario}
\setbeamercolor{palette primary}{bg=colorPrimario,fg=white}
\setbeamercolor{palette secondary}{bg=colorSecundario,fg=white}
\setbeamercolor{palette tertiary}{bg=colorAccento,fg=white}
\setbeamercolor{block title}{bg=colorPrimario!90,fg=white}
\setbeamercolor{block body}{bg=colorPrimario!10,fg=black}
\setbeamercolor{block title example}{bg=colorAccento!90,fg=white}
\setbeamercolor{block body example}{bg=colorAccento!10,fg=black}
\setbeamercolor{block title alerted}{bg=colorSecundario!90,fg=white}
\setbeamercolor{block body alerted}{bg=colorSecundario!10,fg=black}

% Plantillas personalizadas
\setbeamertemplate{navigation symbols}{}
\setbeamertemplate{footline}[frame number]
\setbeamertemplate{itemize items}[circle]
\setbeamertemplate{enumerate items}[circle]
\setbeamerfont{title}{size=\huge,series=\bfseries}
\setbeamerfont{frametitle}{size=\Large,series=\bfseries}

% Comandos personalizados para énfasis
\renewcommand{\alert}[1]{\textcolor{colorSecundario}{\textbf{#1}}}
\newcommand{\highlight}[1]{\textcolor{colorPrimario}{\textbf{#1}}}
\newcommand{\importante}[1]{\textcolor{colorAdvertencia}{\textbf{#1}}}
\newcommand{\exito}[1]{\textcolor{colorAccento}{\textbf{#1}}}

% Variables del documento (serán reemplazadas por Jinja2)
\title{Introducción a la Probabilidad y Estadística}
\subtitle{¡Una aventura con números y adivinanzas!}
\author{Probabilidad y Estadística}
\institute{Aprendiendo con diversión 🎓}
\date{\today}

\begin{document}

% Página de título con diseño atractivo
\begin{frame}[plain]
\begin{tikzpicture}[remember picture,overlay]
  % Fondo decorativo
  \fill[colorPrimario!20] (current page.south west) rectangle (current page.north east);
  \fill[colorAccento!30] (current page.south west) -- (current page.south east) -- 
        ([yshift=-3cm]current page.north east) -- ([yshift=-3cm]current page.north west) -- cycle;
\end{tikzpicture}
\titlepage
\end{frame}

% Diapositivas generadas dinámicamente
\begin{frame}[fragile]{¿Qué vamos a aprender?}
\begin{alertblock}{💡 Nota importante}
¡Bienvenidos, exploradores de datos! 
\end{alertblock}

\begin{itemize}
  \item Vamos a descubrir dos superpoderes:
\end{itemize}

\begin{itemize}
  \item  Estadística: El poder de entender lo que ya pasó (¡como un detective!).
\end{itemize}

\begin{itemize}
  \item  Probabilidad: El poder de adivinar lo que podría pasar (¡como un mago!).
\end{itemize}

\begin{exampleblock}{✨ Ejemplo}
Estadística es ver qué sabor de helado fue el más vendido ayer. Probabilidad es adivinar cuál será el más vendido mañana.
\end{exampleblock}

\end{frame}

\begin{frame}[fragile]{¿Qué es la Estadística? ¡El Poder del Detective!}
\begin{itemize}
  \item Es el arte de coleccionar, organizar y entender la información (los datos).
\end{itemize}

\begin{itemize}
  \item Nos ayuda a descubrir secretos escondidos en los números.
\end{itemize}

\begin{alertblock}{💡 Nota importante}
¡Es como encontrar pistas para resolver un misterio! 
\end{alertblock}

\begin{exampleblock}{✨ Ejemplo}
Hacer una encuesta en tu clase para saber cuál es el superhéroe favorito de todos.
\end{exampleblock}

\end{frame}

\begin{frame}[fragile]{La Estadística está en TODAS PARTES}
\begin{itemize}
  \item  En tu celular: viendo cuántos 'likes' tiene una foto.
\end{itemize}

\begin{itemize}
  \item  En los deportes: viendo quién ha metido más goles.
\end{itemize}

\begin{itemize}
  \item  En los videojuegos: viendo tu récord de puntos.
\end{itemize}

\begin{itemize}
  \item  En Netflix: mostrándote las series más populares.
\end{itemize}

\begin{alertblock}{💡 Nota importante}
¡Usas estadísticas todos los días sin darte cuenta!
\end{alertblock}

\end{frame}

\begin{frame}[fragile]{Ejemplo de Estadística: La Guerra de Helados}
\begin{exampleblock}{🎯 Problema}
Preguntamos a 10 amigos su sabor de helado favorito.
\end{exampleblock}

\begin{center}
\begin{tabular}{cc}
\toprule
\rowcolor{colorPrimario!20}
\textbf{Sabor} & \textbf{Nº de Fans} \\
\midrule
\rowcolor{colorFondo}
Chocolate  & 4 \\
Vainilla  & 3 \\
\rowcolor{colorFondo}
Fresa  & 2 \\
Limón  & 1 \\
\bottomrule
\end{tabular}
\end{center}

\begin{center}
\colorbox{colorFondo}{%
  \parbox{0.9\textwidth}{%
    \begin{align*}
    Conclusión del detective: ¡El chocolate es el rey de los sabores!
    \end{align*}
  }%
}
\end{center}

\end{frame}

\begin{frame}[fragile]{¿Qué es la Probabilidad? ¡El Poder del Mago!}
\begin{itemize}
  \item Es el arte de medir las posibilidades de que algo ocurra en el futuro.
\end{itemize}

\begin{itemize}
  \item Se usa cuando hay duda o no estamos seguros de lo que pasará.
\end{itemize}

\begin{alertblock}{💡 Nota importante}
¡Es como predecir el futuro con matemáticas! 
\end{alertblock}

\begin{exampleblock}{✨ Ejemplo}
¿Qué tan posible es que llueva hoy? ¿50%? ¿80%?
\end{exampleblock}

\end{frame}

\begin{frame}[fragile]{La Probabilidad está en JUEGO}
\begin{itemize}
  \item  En los juegos de mesa: al lanzar un dado.
\end{itemize}

\begin{itemize}
  \item  En las cartas: ¿qué chance tienes de sacar un As?
\end{itemize}

\begin{itemize}
  \item  En los deportes: ¿ganará tu equipo el próximo partido?
\end{itemize}

\begin{itemize}
  \item  En el clima: cuando ves el pronóstico del tiempo.
\end{itemize}

\begin{itemize}
  \item  En los sorteos: ¿qué tan posible es que ganes?
\end{itemize}

\begin{alertblock}{💡 Nota importante}
¡La probabilidad te ayuda a apostar y tomar decisiones!
\end{alertblock}

\end{frame}

\begin{frame}[fragile]{Ejemplo de Probabilidad: La Bolsa Misteriosa}
\begin{exampleblock}{🎯 Problema}
Tienes una bolsa con 10 canicas: 7 rojas y 3 azules.
\end{exampleblock}

\begin{itemize}
  \item Si metes la mano sin mirar, ¿qué color es más fácil de sacar?
\end{itemize}

\begin{itemize}
  \item Respuesta: ¡El rojo!
\end{itemize}

\begin{itemize}
  \item Tienes más chances de sacar una canica roja porque hay más.
\end{itemize}

\begin{alertblock}{💡 Nota importante}
¡La probabilidad te dice cuál es la apuesta más segura! 
\end{alertblock}

\end{frame}

\begin{frame}[fragile]{Diferencias Clave: Detective vs. Mago}
\begin{center}
\begin{tabular}{cc}
\toprule
\rowcolor{colorPrimario!20}
\textbf{Estadística (Detective )} & \textbf{Probabilidad (Mago )} \\
\midrule
\rowcolor{colorFondo}
Mira al PASADO (lo que ya ocurrió) & Mira al FUTURO (lo que podría ocurrir) \\
Trabaja con datos SEGUROS & Trabaja con la DUDA y el AZAR \\
\rowcolor{colorFondo}
Ej: Las notas del examen anterior & Ej: La chance de aprobar el próximo examen \\
\bottomrule
\end{tabular}
\end{center}

\begin{alertblock}{💡 Nota importante}
¡Son dos caras de la misma moneda mágica! 
\end{alertblock}

\end{frame}

\begin{frame}[fragile]{¿Y esto para qué sirve?}
\begin{itemize}
  \item  En medicina: para saber si una vacuna funciona.
\end{itemize}

\begin{itemize}
  \item  En negocios: para decidir qué producto vender más.
\end{itemize}

\begin{itemize}
  \item  En ciencia: para descubrir patrones en la naturaleza.
\end{itemize}

\begin{itemize}
  \item ¡En resumen: para tomar decisiones más inteligentes!
\end{itemize}

\begin{alertblock}{💡 Nota importante}
¡Te ayuda a no equivocarte tanto! 
\end{alertblock}

\end{frame}

\begin{frame}[fragile]{Palabras para Empezar}
\begin{itemize}
  \item  Población: TODO el grupo que quieres estudiar (ej: todos los estudiantes de un colegio).
\end{itemize}

\begin{itemize}
  \item  Muestra: Una pequeña parte de ese grupo (ej: solo los estudiantes de tu salón).
\end{itemize}

\begin{itemize}
  \item  Evento: Algo que puede pasar (ej: que salga 'cara' en una moneda).
\end{itemize}

\begin{itemize}
  \item  Experimento: La acción que haces para ver qué pasa (ej: lanzar la moneda).
\end{itemize}

\begin{alertblock}{💡 Nota importante}
¡Estudiamos una muestra para entender a toda la población!
\end{alertblock}

\end{frame}

\begin{frame}[fragile]{¡Un Experimento para Ti!}
\begin{exampleblock}{🎯 Problema}
Lanza una moneda 10 veces y anota los resultados en un papel.
\end{exampleblock}

\begin{itemize}
  \item Estadística (tu lado detective): Cuenta cuántas 'caras' y cuántos 'sellos' te salieron.
\end{itemize}

\begin{itemize}
  \item Probabilidad (tu lado mago): ¿Qué crees que debería salir? Teóricamente, es 50% y 50%.
\end{itemize}

\begin{alertblock}{💡 Nota importante}
¿Tus resultados se parecen a la teoría? ¡A veces sí, a veces no! Eso es el azar.
\end{alertblock}

\end{frame}

\begin{frame}[fragile]{La Escalera de la Probabilidad}
\begin{itemize}
  \item Imposible (0%): Que un perro maúlle.
\end{itemize}

\begin{itemize}
  \item Poco probable (25%): Ganar una rifa con muchos números.
\end{itemize}

\begin{itemize}
  \item Puede ser (50%): Que salga 'sello' al lanzar una moneda.
\end{itemize}

\begin{itemize}
  \item Muy probable (75%): Que un helado se derrita en el sol.
\end{itemize}

\begin{itemize}
  \item Seguro (100%): Que después del 1 venga el 2.
\end{itemize}

\begin{alertblock}{💡 Nota importante}
¡Todo lo que puede pasar tiene un lugar en esta escalera!
\end{alertblock}

\end{frame}

\begin{frame}[fragile]{¿Listo para la Aventura?}
\begin{itemize}
  \item  Ya sabes qué es la Estadística (el detective).
\end{itemize}

\begin{itemize}
  \item  Ya sabes qué es la Probabilidad (el mago).
\end{itemize}

\begin{itemize}
  \item  Viste que están en todos lados.
\end{itemize}

\begin{alertblock}{💡 Nota importante}
¡Ahora estás listo para aprender los trucos de cada uno!
\end{alertblock}

\begin{itemize}
  \item En las próximas clases veremos:
\end{itemize}

\begin{itemize}
  \item  Cómo hacer tablas y gráficos geniales.
\end{itemize}

\begin{itemize}
  \item  Cómo encontrar el 'corazón' de los datos.
\end{itemize}

\begin{itemize}
  \item  Cómo calcular las chances exactas de que algo pase.
\end{itemize}

\end{frame}


% Diapositiva final atractiva
\begin{frame}[plain]
\begin{tikzpicture}[remember picture,overlay]
  \fill[colorAccento!20] (current page.south west) rectangle (current page.north east);
  \node[font=\Huge\bfseries,text=colorPrimario] at (current page.center) {¿Preguntas? 🤔};
  \node[font=\large,text=colorSecundario,below=1.5cm] at (current page.center) {¡Sigue aprendiendo! 🚀};
\end{tikzpicture}
\end{frame}

\end{document}