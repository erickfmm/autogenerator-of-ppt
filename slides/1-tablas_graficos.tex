\documentclass[aspectratio=169]{beamer}

% Tema moderno y lúdico
\usetheme{Boadilla}
\usecolortheme{dolphin}

% Paquetes esenciales
\usepackage[utf8]{inputenc}
\usepackage[spanish]{babel}
\usepackage{amsmath}
\usepackage{amssymb}
\usepackage{tikz}
\usetikzlibrary{shapes, arrows, positioning, calc}
\usepackage{pgfplots}
\pgfplotsset{compat=1.18}
\usepackage{pgf-pie}
\usepackage{booktabs}
\usepackage{xcolor}
\usepackage{colortbl}
\usepackage{newunicodechar}

% Configurar emojis como texto simple
\newunicodechar{🎲}{\textsf{[dado]}}
\newunicodechar{🎯}{\textsf{[objetivo]}}
\newunicodechar{📋}{\textsf{[lista]}}
\newunicodechar{⭐}{\textsf{[estrella]}}
\newunicodechar{🔮}{\textsf{[bola]}}
\newunicodechar{📏}{\textsf{[regla]}}
\newunicodechar{🃏}{\textsf{[carta]}}
\newunicodechar{🔄}{\textsf{[ciclo]}}
\newunicodechar{🔢}{\textsf{[numeros]}}
\newunicodechar{➕}{\textsf{[+]}}
\newunicodechar{✖}{\textsf{[x]}}
\newunicodechar{🔗}{\textsf{[cadena]}}
\newunicodechar{🪙}{\textsf{[moneda]}}
\newunicodechar{🔑}{\textsf{[llave]}}
\newunicodechar{🔍}{\textsf{[lupa]}}
\newunicodechar{📊}{\textsf{[grafico]}}
\newunicodechar{🎴}{\textsf{[cartas]}}
\newunicodechar{💪}{\textsf{[fuerza]}}
\newunicodechar{🌟}{\textsf{[estrella]}}
\newunicodechar{💡}{\textsf{[idea]}}
\newunicodechar{✨}{\textsf{[brillos]}}
\newunicodechar{✅}{\textsf{[check]}}
\newunicodechar{🎓}{\textsf{[gorro]}}
\newunicodechar{🤔}{\textsf{[pensar]}}
\newunicodechar{🚀}{\textsf{[cohete]}}
\newunicodechar{♥}{\ensuremath{\heartsuit}}
\newunicodechar{₁}{\textsubscript{1}}
\newunicodechar{₂}{\textsubscript{2}}
\newunicodechar{✖}{\textsf{x}}
\newunicodechar{✓}{\textsf{[ok]}}
\newunicodechar{⚫}{\textsf{[punto]}}
\newunicodechar{👥}{\textsf{[grupo]}}
\newunicodechar{️}{}

% Colores personalizados vibrantes y lúdicos
\definecolor{colorPrimario}{RGB}{41, 128, 185}      % Azul vibrante
\definecolor{colorSecundario}{RGB}{231, 76, 60}     % Rojo coral
\definecolor{colorAccento}{RGB}{46, 204, 113}       % Verde esmeralda
\definecolor{colorAdvertencia}{RGB}{241, 196, 15}   % Amarillo dorado
\definecolor{colorMorado}{RGB}{155, 89, 182}        % Morado amigable
\definecolor{colorNaranja}{RGB}{230, 126, 34}       % Naranja cálido
\definecolor{colorFondo}{RGB}{236, 240, 241}        % Gris claro de fondo

% Configuración del tema
\setbeamercolor{structure}{fg=colorPrimario}
\setbeamercolor{palette primary}{bg=colorPrimario,fg=white}
\setbeamercolor{palette secondary}{bg=colorSecundario,fg=white}
\setbeamercolor{palette tertiary}{bg=colorAccento,fg=white}
\setbeamercolor{block title}{bg=colorPrimario!90,fg=white}
\setbeamercolor{block body}{bg=colorPrimario!10,fg=black}
\setbeamercolor{block title example}{bg=colorAccento!90,fg=white}
\setbeamercolor{block body example}{bg=colorAccento!10,fg=black}
\setbeamercolor{block title alerted}{bg=colorSecundario!90,fg=white}
\setbeamercolor{block body alerted}{bg=colorSecundario!10,fg=black}

% Plantillas personalizadas
\setbeamertemplate{navigation symbols}{}
\setbeamertemplate{footline}[frame number]
\setbeamertemplate{itemize items}[circle]
\setbeamertemplate{enumerate items}[circle]
\setbeamerfont{title}{size=\huge,series=\bfseries}
\setbeamerfont{frametitle}{size=\Large,series=\bfseries}

% Comandos personalizados para énfasis
\renewcommand{\alert}[1]{\textcolor{colorSecundario}{\textbf{#1}}}
\newcommand{\highlight}[1]{\textcolor{colorPrimario}{\textbf{#1}}}
\newcommand{\importante}[1]{\textcolor{colorAdvertencia}{\textbf{#1}}}
\newcommand{\exito}[1]{\textcolor{colorAccento}{\textbf{#1}}}

% Variables del documento (serán reemplazadas por Jinja2)
\title{Representación de datos a través de tablas y gráficos}
\subtitle{¡Organicemos los datos de forma divertida! }
\author{Probabilidad y Estadística}
\institute{Aprendiendo con diversión 🎓}
\date{\today}

\begin{document}

% Página de título con diseño atractivo
\begin{frame}[plain]
\begin{tikzpicture}[remember picture,overlay]
  % Fondo decorativo
  \fill[colorPrimario!20] (current page.south west) rectangle (current page.north east);
  \fill[colorAccento!30] (current page.south west) -- (current page.south east) -- 
        ([yshift=-3cm]current page.north east) -- ([yshift=-3cm]current page.north west) -- cycle;
\end{tikzpicture}
\titlepage
\end{frame}

% Diapositivas generadas dinámicamente
\begin{frame}[fragile]{¿Por qué organizar datos?}
\begin{alertblock}{💡 Nota importante}
¡Imagina tu cuarto desordenado vs ordenado! 
\end{alertblock}

\begin{itemize}
  \item Datos desordenados = confusión 
\end{itemize}

\begin{itemize}
  \item Datos organizados = claridad y respuestas 
\end{itemize}

\begin{exampleblock}{✨ Ejemplo}
¿Qué helado prefieren tus amigos? Con una tabla lo sabrás al instante
\end{exampleblock}

\end{frame}

\begin{frame}[fragile]{¿Qué son los datos?}
\begin{itemize}
  \item Son información que recolectamos
\end{itemize}

\begin{itemize}
  \item Pueden ser números: edades, pesos, calificaciones
\end{itemize}

\begin{itemize}
  \item O categorías: colores, sabores, deportes
\end{itemize}

\begin{exampleblock}{✨ Ejemplo}
Datos de tu clase: alturas de todos los estudiantes
\end{exampleblock}

\begin{alertblock}{💡 Nota importante}
¡Los datos cuentan historias, solo hay que saber leerlos! 
\end{alertblock}

\end{frame}

\begin{frame}[fragile]{Tablas de Frecuencia - ¿Qué son?}
\begin{itemize}
  \item Es como hacer un conteo organizado 
\end{itemize}

\begin{itemize}
  \item Agrupamos datos iguales y contamos
\end{itemize}

\begin{alertblock}{💡 Nota importante}
¡Es como ordenar tu colección de cartas por tipo!
\end{alertblock}

\begin{exampleblock}{✨ Ejemplo}
Si tienes muchas manzanas rojas y pocas verdes, la tabla lo muestra claramente
\end{exampleblock}

\end{frame}

\begin{frame}[fragile]{Frecuencia Absoluta}
\begin{itemize}
  \item Es simplemente: ¿cuántas veces aparece?
\end{itemize}

\begin{itemize}
  \item Es el conteo directo, el número total
\end{itemize}

\begin{alertblock}{💡 Nota importante}
¡Es como contar con los dedos! 
\end{alertblock}

\begin{exampleblock}{✨ Ejemplo}
Colores de autos en el estacionamiento:
\end{exampleblock}

\begin{itemize}
  \item Rojos: 5, Azules: 3, Blancos: 7
\end{itemize}

\begin{itemize}
  \item Las frecuencias absolutas son: 5, 3 y 7
\end{itemize}

\end{frame}

\begin{frame}[fragile]{Frecuencia Absoluta - Visualización}
\begin{center}
\begin{tikzpicture}
  \begin{axis}[
    ybar,
    width=0.8\textwidth,
    height=0.45\textheight,
    bar width=15pt,
    ylabel={Cantidad},
    xlabel={Color de auto},
    symbolic x coords={Rojos,Azules,Blancos},
    xtick=data,
    x tick label style={rotate=45, anchor=east},
    ymin=0,
    enlarge x limits=0.15,
    legend style={at={(0.5,-0.25)}, anchor=north, legend columns=-1},
    nodes near coords,
    every node near coord/.append style={font=\footnotesize},
    grid=major,
    ymajorgrids=true,
    grid style={dashed,gray!30}
  ]
  \addplot[fill=colorPrimario!70] coordinates {
    (Rojos,5)
    (Azules,3)
    (Blancos,7)
  };
  \legend{Frecuencia}
  \end{axis}
\end{tikzpicture}
\end{center}
\end{frame}

\begin{frame}[fragile]{Frecuencia Relativa - ¿Qué parte del todo?}
\begin{itemize}
  \item Es la proporción, ¡el pedazo del pastel!
\end{itemize}

\begin{itemize}
  \item Responde: ¿qué parte del total representa cada categoría?
\end{itemize}

\begin{center}
\colorbox{colorFondo}{%
  \parbox{0.8\textwidth}{%
    \begin{center}
    \Large\color{colorPrimario}
    $Frec. Relativa = (Las veces que aparece) / (El total de datos)$
    \end{center}
  }%
}
\end{center}

\begin{alertblock}{💡 Nota importante}
¡Es como saber qué porción de pizza te toca! 
\end{alertblock}

\begin{exampleblock}{✨ Ejemplo}
Si hay 10 autos y 5 son rojos: 5 de 10 = la mitad = 50 por ciento
\end{exampleblock}

\end{frame}

\begin{frame}[fragile]{Ejemplo: ¡Guerra de Notas!}
\begin{exampleblock}{🎯 Problema}
Notas de 6 estudiantes: 15, 18, 15, 20, 18, 15
\end{exampleblock}

\begin{itemize}
  \item Paso 1: ¿Qué notas distintas hay? (15, 18, 20)
\end{itemize}

\begin{itemize}
  \item Paso 2: Contar cuántos sacaron cada nota (Frec. Absoluta)
\end{itemize}

\begin{itemize}
  \item Paso 3: Ver qué parte del grupo sacó esa nota (Frec. Relativa)
\end{itemize}

\begin{alertblock}{💡 Nota importante}
¡Vamos a construir la tabla juntos!
\end{alertblock}

\end{frame}

\begin{frame}[fragile]{Tabla de Notas del Ejemplo}
\begin{center}
\begin{tabular}{cc}
\toprule
\rowcolor{colorPrimario!20}
\textbf{Nota} & \textbf{Frecuencia Absoluta} \\
\midrule
\rowcolor{colorFondo}
15 & 3 \\
18 & 2 \\
\rowcolor{colorFondo}
20 & 1 \\
\bottomrule
\end{tabular}
\end{center}

\begin{itemize}
  \item Total de estudiantes: 6
\end{itemize}

\begin{center}
\begin{tabular}{cc}
\toprule
\rowcolor{colorPrimario!20}
\textbf{Nota} & \textbf{Frecuencia Relativa} \\
\midrule
\rowcolor{colorFondo}
15 & 3 de 6 = 50 por ciento \\
18 & 2 de 6 = 33 por ciento \\
\rowcolor{colorFondo}
20 & 1 de 6 = 17 por ciento \\
\bottomrule
\end{tabular}
\end{center}

\begin{center}
\colorbox{colorFondo}{%
  \parbox{0.9\textwidth}{%
    \begin{align*}
    ¡La nota 15 es la más común! La mitad de la clase la sacó.
    \end{align*}
  }%
}
\end{center}

\begin{alertblock}{💡 Nota importante}
¡Si sumas los porcentajes, te da el 100 por ciento de la clase!
\end{alertblock}

\end{frame}

\begin{frame}[fragile]{Gráfico del Ejemplo - Notas}
\begin{center}
\begin{tikzpicture}
  \begin{axis}[
    ybar,
    width=0.8\textwidth,
    height=0.45\textheight,
    bar width=15pt,
    ylabel={Nº de Estudiantes},
    xlabel={Calificaciones},
    symbolic x coords={Nota 15,Nota 18,Nota 20},
    xtick=data,
    x tick label style={rotate=45, anchor=east},
    ymin=0,
    enlarge x limits=0.15,
    legend style={at={(0.5,-0.25)}, anchor=north, legend columns=-1},
    nodes near coords,
    every node near coord/.append style={font=\footnotesize},
    grid=major,
    ymajorgrids=true,
    grid style={dashed,gray!30}
  ]
  \addplot[fill=colorPrimario!70] coordinates {
    (Nota 15,3)
    (Nota 18,2)
    (Nota 20,1)
  };
  \legend{Estudiantes}
  \end{axis}
\end{tikzpicture}
\end{center}
\end{frame}

\begin{frame}[fragile]{Gráfico de Barras }
\begin{itemize}
  \item ¡Compara cosas usando barras de diferentes alturas!
\end{itemize}

\begin{itemize}
  \item Cada barra es una categoría (ej: sabor de helado).
\end{itemize}

\begin{itemize}
  \item La altura de la barra te dice cuántos hay de esa categoría.
\end{itemize}

\begin{alertblock}{💡 Nota importante}
¡Perfecto para ver quién gana en una encuesta!
\end{alertblock}

\begin{exampleblock}{✨ Ejemplo}
Deportes favoritos: Fútbol (10), Básquet (7), Vóley (5)
\end{exampleblock}

\begin{itemize}
  \item ¡La barra de Fútbol será la más alta! 
\end{itemize}

\end{frame}

\begin{frame}[fragile]{Ejemplo - Deportes Favoritos}
\begin{center}
\begin{tikzpicture}
  \begin{axis}[
    ybar,
    width=0.8\textwidth,
    height=0.45\textheight,
    bar width=30pt,
    ylabel={Votos de los fans},
    xlabel={Deporte},
    symbolic x coords={Fútbol,Básquet,Vóley},
    xtick=data,
    x tick label style={rotate=45, anchor=east},
    ymin=0,
    enlarge x limits=0.15,
    legend style={at={(0.5,-0.25)}, anchor=north, legend columns=-1},
    nodes near coords,
    every node near coord/.append style={font=\footnotesize},
    grid=major,
    ymajorgrids=true,
    grid style={dashed,gray!30}
  ]
  \addplot[fill=colorPrimario!70] coordinates {
    (Fútbol,10)
    (Básquet,7)
    (Vóley,5)
  };
  \legend{Preferencias}
  \end{axis}
\end{tikzpicture}
\end{center}
\end{frame}

\begin{frame}[fragile]{Histograma }
\begin{itemize}
  \item ¡Parecido al gráfico de barras, pero para números!
\end{itemize}

\begin{itemize}
  \item Agrupa números en rangos (ej: edades de 0-10, 11-20...).
\end{itemize}

\begin{itemize}
  \item Las barras van pegaditas para mostrar continuidad.
\end{itemize}

\begin{exampleblock}{✨ Ejemplo}
Alturas de personas: 1.50-1.60m, 1.60-1.70m, etc.
\end{exampleblock}

\begin{alertblock}{💡 Nota importante}
¡Muestra dónde se amontonan más los datos!
\end{alertblock}

\end{frame}

\begin{frame}[fragile]{Gráfico Circular (o de Pastel )}
\begin{itemize}
  \item ¡Un círculo que muestra las partes de un todo!
\end{itemize}

\begin{itemize}
  \item Cada 'rebanada' es una categoría.
\end{itemize}

\begin{itemize}
  \item El tamaño de la rebanada te dice qué tan grande es esa parte.
\end{itemize}

\begin{alertblock}{💡 Nota importante}
¡Ideal para ver cómo se reparte algo al 100 por ciento!
\end{alertblock}

\begin{exampleblock}{✨ Ejemplo}
En tu celular: 50 por ciento para apps, 30 por ciento para fotos, 20 por ciento para música.
\end{exampleblock}

\end{frame}

\begin{frame}[fragile]{Ejemplo - Uso de Memoria del Celular}
\begin{center}
\begin{tikzpicture}
  \pie[
    radius=3,
    text=legend,
    color={colorPrimario!70, colorAccento!70, colorSecundario!70, colorAdvertencia!70, colorMorado!70, colorNaranja!70}
  ]{
    50/Apps,    30/Fotos,    20/Música  }
\end{tikzpicture}
\end{center}
\end{frame}

\begin{frame}[fragile]{Gráfico de Líneas }
\begin{itemize}
  \item ¡Conecta puntos para mostrar cómo algo cambia con el tiempo!
\end{itemize}

\begin{itemize}
  \item Perfecto para ver tendencias: ¿sube, baja o se mantiene?
\end{itemize}

\begin{exampleblock}{✨ Ejemplo}
La temperatura durante la semana: ¿hizo más calor el lunes o el viernes?
\end{exampleblock}

\begin{alertblock}{💡 Nota importante}
¡La línea te cuenta una historia de subidas y bajadas! 
\end{alertblock}

\end{frame}

\begin{frame}[fragile]{Ejemplo - Temperatura de la Semana}
\begin{center}
\begin{tikzpicture}
  \begin{axis}[
    width=0.85\textwidth,
    height=0.6\textheight,
    ylabel={Temperatura (°C)},
    xlabel={Día de la Semana},
    grid=both,
    major grid style={line width=.2pt,draw=gray!50},
    minor grid style={line width=.1pt,draw=gray!20},
    legend pos=north east,
    mark size=3pt,
    line width=2pt
  ]
  \addplot[color=colorPrimario, mark=*] coordinates {
    (1,22)
    (2,24)
    (3,21)
    (4,23)
    (5,25)
    (6,20)
    (7,22)
  };
  \legend{Temperatura}
  \end{axis}
\end{tikzpicture}
\end{center}
\end{frame}

\begin{frame}[fragile]{¿Qué Gráfico Uso? ¡Elige tu arma!}
\begin{center}
\begin{tabular}{cc}
\toprule
\rowcolor{colorPrimario!20}
\textbf{Si quieres...} & \textbf{Usa...} \\
\midrule
\rowcolor{colorFondo}
Comparar categorías & Gráfico de Barras \\
Distribuir números & Histograma \\
\rowcolor{colorFondo}
Partes de un todo & Gráfico Circular \\
Cambios en el tiempo & Gráfico de Líneas \\
\bottomrule
\end{tabular}
\end{center}

\begin{alertblock}{💡 Nota importante}
¡Elige el mejor gráfico para contar tu historia! 
\end{alertblock}

\end{frame}

\begin{frame}[fragile]{El Promedio - El 'Valor Justo'}
\begin{itemize}
  \item También llamado 'media'. ¡Es el punto de equilibrio! 
\end{itemize}

\begin{itemize}
  \item Es el número que representa a todo el grupo.
\end{itemize}

\begin{alertblock}{💡 Nota importante}
¡Imagina repartir todos los caramelos en partes iguales!
\end{alertblock}

\begin{exampleblock}{✨ Ejemplo}
Si 3 amigos tienen 2, 4 y 6 caramelos, el promedio es 4 para cada uno.
\end{exampleblock}

\end{frame}

\begin{frame}[fragile]{¿Cómo Calcular el Promedio?}
\begin{itemize}
  \item Paso 1: ¡Suma todos los datos!
\end{itemize}

\begin{itemize}
  \item Paso 2: ¡Divide por la cantidad de datos que sumaste!
\end{itemize}

\begin{center}
\colorbox{colorFondo}{%
  \parbox{0.8\textwidth}{%
    \begin{center}
    \Large\color{colorPrimario}
    $Promedio = (Suma de todos los datos) / (Cuántos datos hay)$
    \end{center}
  }%
}
\end{center}

\begin{alertblock}{💡 Nota importante}
¡Es como compartir la cuenta entre amigos! 
\end{alertblock}

\end{frame}

\begin{frame}[fragile]{Ejemplo de Promedio}
\begin{exampleblock}{✨ Ejemplo}
Tus notas: 10, 12, 17
\end{exampleblock}

\begin{itemize}
  \item Paso 1: Sumar todo
\end{itemize}

\begin{center}
\colorbox{colorFondo}{%
  \parbox{0.9\textwidth}{%
    \begin{align*}
    10 + 12 + 17 = 39
    \end{align*}
  }%
}
\end{center}

\begin{itemize}
  \item Paso 2: Dividir entre 3 (porque son 3 notas)
\end{itemize}

\begin{center}
\colorbox{colorFondo}{%
  \parbox{0.9\textwidth}{%
    \begin{align*}
    Promedio = 39 / 3 = 13
    \end{align*}
  }%
}
\end{center}

\begin{alertblock}{💡 Nota importante}
¡Tu promedio es 13! Ese es el 'resumen' de tus notas.
\end{alertblock}

\end{frame}

\begin{frame}[fragile]{Practicando con Frutas }
\begin{exampleblock}{🎯 Problema}
Ventas de manzanas por día: Lunes (10), Martes (15), Miércoles (20)
\end{exampleblock}

\begin{itemize}
  \item ¿Cuántas manzanas se venden en promedio cada día?
\end{itemize}

\end{frame}

\begin{frame}[fragile]{Visualización - Ventas de Manzanas}
\begin{center}
\begin{tikzpicture}
  \begin{axis}[
    ybar,
    width=0.8\textwidth,
    height=0.45\textheight,
    bar width=15pt,
    ylabel={Manzanas vendidas},
    xlabel={Día de la semana},
    symbolic x coords={Lunes,Martes,Miércoles},
    xtick=data,
    x tick label style={rotate=45, anchor=east},
    ymin=0,
    enlarge x limits=0.15,
    legend style={at={(0.5,-0.25)}, anchor=north, legend columns=-1},
    nodes near coords,
    every node near coord/.append style={font=\footnotesize},
    grid=major,
    ymajorgrids=true,
    grid style={dashed,gray!30}
  ]
  \addplot[fill=colorPrimario!70] coordinates {
    (Lunes,10)
    (Martes,15)
    (Miércoles,20)
  };
  \legend{Ventas}
  \end{axis}
\end{tikzpicture}
\end{center}
\end{frame}

\begin{frame}[fragile]{Solución - Promedio de Manzanas}
\begin{center}
\colorbox{colorFondo}{%
  \parbox{0.9\textwidth}{%
    \begin{align*}
    Suma: 10 + 15 + 20 = 45
    \end{align*}
  }%
}
\end{center}

\begin{center}
\colorbox{colorFondo}{%
  \parbox{0.9\textwidth}{%
    \begin{align*}
    Promedio: 45 / 3 días = 15 manzanas/día
    \end{align*}
  }%
}
\end{center}

\begin{itemize}
  \item ¡En promedio, se venden 15 manzanas al día! 
\end{itemize}

\end{frame}

\begin{frame}[fragile]{Ejercicio: ¡El Detective del Clima! }
\begin{exampleblock}{🎯 Problema}
Temperaturas de 5 días (°C): 20, 25, 22, 28, 20
\end{exampleblock}

\begin{itemize}
  \item Tarea 1: ¿Cuál es la temperatura más común?
\end{itemize}

\begin{itemize}
  \item Tarea 2: Calcula la temperatura promedio.
\end{itemize}

\begin{itemize}
  \item Tarea 3: ¿Qué gráfico usarías para ver la evolución del clima?
\end{itemize}

\begin{alertblock}{💡 Nota importante}
¡Piensa qué historia quieres contar con los datos!
\end{alertblock}

\end{frame}

\begin{frame}[fragile]{Solución - Tabla}
\begin{center}
\begin{tabular}{cc}
\toprule
\rowcolor{colorPrimario!20}
\textbf{Temperatura} & \textbf{Días} \\
\midrule
\rowcolor{colorFondo}
20 grados C & 2 \\
22 grados C & 1 \\
\rowcolor{colorFondo}
25 grados C & 1 \\
28 grados C & 1 \\
\bottomrule
\end{tabular}
\end{center}

\begin{alertblock}{💡 Nota importante}
¡La temperatura más común (la moda) es 20 grados C!
\end{alertblock}

\end{frame}

\begin{frame}[fragile]{Gráfico - Distribución de Temperaturas}
\begin{center}
\begin{tikzpicture}
  \begin{axis}[
    ybar,
    width=0.8\textwidth,
    height=0.45\textheight,
    bar width=15pt,
    ylabel={Frecuencia (días)},
    xlabel={Temperatura},
    symbolic x coords={20°C,22°C,25°C,28°C},
    xtick=data,
    x tick label style={rotate=45, anchor=east},
    ymin=0,
    enlarge x limits=0.15,
    legend style={at={(0.5,-0.25)}, anchor=north, legend columns=-1},
    nodes near coords,
    every node near coord/.append style={font=\footnotesize},
    grid=major,
    ymajorgrids=true,
    grid style={dashed,gray!30}
  ]
  \addplot[fill=colorPrimario!70] coordinates {
    (20°C,2)
    (22°C,1)
    (25°C,1)
    (28°C,1)
  };
  \legend{Distribución}
  \end{axis}
\end{tikzpicture}
\end{center}
\end{frame}

\begin{frame}[fragile]{Solución - Promedio}
\begin{center}
\colorbox{colorFondo}{%
  \parbox{0.9\textwidth}{%
    \begin{align*}
    Suma: 20 + 25 + 22 + 28 + 20 = 115
    \end{align*}
  }%
}
\end{center}

\begin{center}
\colorbox{colorFondo}{%
  \parbox{0.9\textwidth}{%
    \begin{align*}
    Promedio: 115 / 5 días = 23°C
    \end{align*}
  }%
}
\end{center}

\begin{itemize}
  \item Interpretación:
\end{itemize}

\begin{itemize}
  \item La temperatura promedio de la semana fue 23°C.
\end{itemize}

\begin{alertblock}{💡 Nota importante}
¡Una semana bastante agradable! 
\end{alertblock}

\end{frame}

\begin{frame}[fragile]{Solución - Gráfico Ideal}
\begin{itemize}
  \item Mejor opción: ¡Gráfico de líneas! 
\end{itemize}

\begin{itemize}
  \item ¿Por qué?
\end{itemize}

\begin{itemize}
  \item  Muestra los cambios día a día.
\end{itemize}

\begin{itemize}
  \item  Podemos ver si el clima mejoró o empeoró.
\end{itemize}

\begin{itemize}
  \item  ¡Es perfecto para ver tendencias en el tiempo!
\end{itemize}

\begin{alertblock}{💡 Nota importante}
Un gráfico de barras también serviría para comparar los días.
\end{alertblock}

\end{frame}

\begin{frame}[fragile]{¡Resumen de la Misión!}
\begin{itemize}
  \item  Misión 1: Organizar datos con Tablas de Frecuencia.
\end{itemize}

\begin{itemize}
  \item  Misión 2: Visualizar datos con Gráficos (barras, pastel, líneas).
\end{itemize}

\begin{itemize}
  \item  Misión 3: Calcular el Promedio para encontrar el 'valor justo'.
\end{itemize}

\begin{itemize}
  \item  ¡Ahora eres un detective de datos capaz de organizarlos y entenderlos! 
\end{itemize}

\end{frame}


% Diapositiva final atractiva
\begin{frame}[plain]
\begin{tikzpicture}[remember picture,overlay]
  \fill[colorAccento!20] (current page.south west) rectangle (current page.north east);
  \node[font=\Huge\bfseries,text=colorPrimario] at (current page.center) {¿Preguntas? 🤔};
  \node[font=\large,text=colorSecundario,below=1.5cm] at (current page.center) {¡Sigue aprendiendo! 🚀};
\end{tikzpicture}
\end{frame}

\end{document}