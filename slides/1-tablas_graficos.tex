\documentclass[aspectratio=169]{beamer}

% Tema moderno y lúdico
\usetheme{Boadilla}
\usecolortheme{dolphin}

% Paquetes esenciales
\usepackage[utf8]{inputenc}
\usepackage[spanish]{babel}
\usepackage{amsmath}
\usepackage{amssymb}
\usepackage{tikz}
\usetikzlibrary{shapes, arrows, positioning, calc}
\usepackage{pgfplots}
\pgfplotsset{compat=1.18}
\usepackage{pgf-pie}
\usepackage{booktabs}
\usepackage{xcolor}
\usepackage{colortbl}
\usepackage{newunicodechar}

% Configurar emojis como texto simple
\newunicodechar{🎲}{\textsf{[dado]}}
\newunicodechar{🎯}{\textsf{[objetivo]}}
\newunicodechar{📋}{\textsf{[lista]}}
\newunicodechar{⭐}{\textsf{[estrella]}}
\newunicodechar{🔮}{\textsf{[bola]}}
\newunicodechar{📏}{\textsf{[regla]}}
\newunicodechar{🃏}{\textsf{[carta]}}
\newunicodechar{🔄}{\textsf{[ciclo]}}
\newunicodechar{🔢}{\textsf{[numeros]}}
\newunicodechar{➕}{\textsf{[+]}}
\newunicodechar{✖}{\textsf{[x]}}
\newunicodechar{🔗}{\textsf{[cadena]}}
\newunicodechar{🪙}{\textsf{[moneda]}}
\newunicodechar{🔑}{\textsf{[llave]}}
\newunicodechar{🔍}{\textsf{[lupa]}}
\newunicodechar{📊}{\textsf{[grafico]}}
\newunicodechar{🎴}{\textsf{[cartas]}}
\newunicodechar{💪}{\textsf{[fuerza]}}
\newunicodechar{🌟}{\textsf{[estrella]}}
\newunicodechar{💡}{\textsf{[idea]}}
\newunicodechar{✨}{\textsf{[brillos]}}
\newunicodechar{✅}{\textsf{[check]}}
\newunicodechar{🎓}{\textsf{[gorro]}}
\newunicodechar{🤔}{\textsf{[pensar]}}
\newunicodechar{🚀}{\textsf{[cohete]}}
\newunicodechar{♥}{\ensuremath{\heartsuit}}
\newunicodechar{₁}{\textsubscript{1}}
\newunicodechar{₂}{\textsubscript{2}}
\newunicodechar{✖}{\textsf{x}}
\newunicodechar{✓}{\textsf{[ok]}}
\newunicodechar{⚫}{\textsf{[punto]}}
\newunicodechar{👥}{\textsf{[grupo]}}
\newunicodechar{️}{}

% Colores personalizados vibrantes y lúdicos
\definecolor{colorPrimario}{RGB}{41, 128, 185}      % Azul vibrante
\definecolor{colorSecundario}{RGB}{231, 76, 60}     % Rojo coral
\definecolor{colorAccento}{RGB}{46, 204, 113}       % Verde esmeralda
\definecolor{colorAdvertencia}{RGB}{241, 196, 15}   % Amarillo dorado
\definecolor{colorMorado}{RGB}{155, 89, 182}        % Morado amigable
\definecolor{colorNaranja}{RGB}{230, 126, 34}       % Naranja cálido
\definecolor{colorFondo}{RGB}{236, 240, 241}        % Gris claro de fondo

% Configuración del tema
\setbeamercolor{structure}{fg=colorPrimario}
\setbeamercolor{palette primary}{bg=colorPrimario,fg=white}
\setbeamercolor{palette secondary}{bg=colorSecundario,fg=white}
\setbeamercolor{palette tertiary}{bg=colorAccento,fg=white}
\setbeamercolor{block title}{bg=colorPrimario!90,fg=white}
\setbeamercolor{block body}{bg=colorPrimario!10,fg=black}
\setbeamercolor{block title example}{bg=colorAccento!90,fg=white}
\setbeamercolor{block body example}{bg=colorAccento!10,fg=black}
\setbeamercolor{block title alerted}{bg=colorSecundario!90,fg=white}
\setbeamercolor{block body alerted}{bg=colorSecundario!10,fg=black}

% Plantillas personalizadas
\setbeamertemplate{navigation symbols}{}
\setbeamertemplate{footline}[frame number]
\setbeamertemplate{itemize items}[circle]
\setbeamertemplate{enumerate items}[circle]
\setbeamerfont{title}{size=\huge,series=\bfseries}
\setbeamerfont{frametitle}{size=\Large,series=\bfseries}

% Comandos personalizados para énfasis
\renewcommand{\alert}[1]{\textcolor{colorSecundario}{\textbf{#1}}}
\newcommand{\highlight}[1]{\textcolor{colorPrimario}{\textbf{#1}}}
\newcommand{\importante}[1]{\textcolor{colorAdvertencia}{\textbf{#1}}}
\newcommand{\exito}[1]{\textcolor{colorAccento}{\textbf{#1}}}

% Variables del documento (serán reemplazadas por Jinja2)
\title{Representación de datos a través de tablas y gráficos}
\subtitle{¡Organicemos los datos de forma divertida! 📊}
\author{Probabilidad y Estadística}
\institute{Aprendiendo con diversión 🎓}
\date{\today}

\begin{document}

% Página de título con diseño atractivo
\begin{frame}[plain]
\begin{tikzpicture}[remember picture,overlay]
  % Fondo decorativo
  \fill[colorPrimario!20] (current page.south west) rectangle (current page.north east);
  \fill[colorAccento!30] (current page.south west) -- (current page.south east) -- 
        ([yshift=-3cm]current page.north east) -- ([yshift=-3cm]current page.north west) -- cycle;
\end{tikzpicture}
\titlepage
\end{frame}

% Diapositivas generadas dinámicamente
\begin{frame}[fragile]{¿Por qué organizar datos?}
\begin{alertblock}{💡 Nota importante}
¡Imagina tu cuarto desordenado vs ordenado! 🏠
\end{alertblock}

\begin{itemize}
  \item Datos desordenados = confusión 😵
\end{itemize}

\begin{itemize}
  \item Datos organizados = claridad y respuestas ✨
\end{itemize}

\begin{exampleblock}{✨ Ejemplo}
¿Qué helado prefieren tus amigos? Con una tabla lo sabrás al instante
\end{exampleblock}

\end{frame}

\begin{frame}[fragile]{¿Qué son los datos?}
\begin{itemize}
  \item Son información que recolectamos
\end{itemize}

\begin{itemize}
  \item Pueden ser números: edades, pesos, calificaciones
\end{itemize}

\begin{itemize}
  \item O categorías: colores, sabores, deportes
\end{itemize}

\begin{exampleblock}{✨ Ejemplo}
Datos de tu clase: alturas de todos los estudiantes
\end{exampleblock}

\begin{alertblock}{💡 Nota importante}
¡Los datos cuentan historias, solo hay que saber leerlos! 📖
\end{alertblock}

\end{frame}

\begin{frame}[fragile]{Tablas de Frecuencia - ¿Qué son?}
\begin{itemize}
  \item Es como hacer un conteo organizado 📝
\end{itemize}

\begin{itemize}
  \item Agrupamos datos iguales y contamos
\end{itemize}

\begin{alertblock}{💡 Nota importante}
¡Es como ordenar tu colección de cartas por tipo!
\end{alertblock}

\begin{exampleblock}{✨ Ejemplo}
Si tienes muchas manzanas rojas y pocas verdes, la tabla lo muestra claramente
\end{exampleblock}

\end{frame}

\begin{frame}[fragile]{Frecuencia Absoluta}
\begin{itemize}
  \item Es simplemente: ¿cuántas veces aparece?
\end{itemize}

\begin{itemize}
  \item Es el conteo directo, el número total
\end{itemize}

\begin{alertblock}{💡 Nota importante}
¡Es como contar con los dedos! ✋
\end{alertblock}

\begin{exampleblock}{✨ Ejemplo}
Colores de autos en el estacionamiento:
\end{exampleblock}

\begin{itemize}
  \item Rojos: 5, Azules: 3, Blancos: 7
\end{itemize}

\begin{itemize}
  \item Las frecuencias absolutas son: 5, 3 y 7
\end{itemize}

\end{frame}

\begin{frame}[fragile]{Frecuencia Absoluta - Visualización}
\begin{center}
\begin{tikzpicture}
  \begin{axis}[
    ybar,
    width=0.8\textwidth,
    height=0.45\textheight,
    bar width=15pt,
    ylabel={Cantidad},
    xlabel={Color de auto},
    symbolic x coords={Rojos,Azules,Blancos},
    xtick=data,
    x tick label style={rotate=45, anchor=east},
    ymin=0,
    enlarge x limits=0.15,
    legend style={at={(0.5,-0.25)}, anchor=north, legend columns=-1},
    nodes near coords,
    every node near coord/.append style={font=\footnotesize},
    grid=major,
    ymajorgrids=true,
    grid style={dashed,gray!30}
  ]
  \addplot[fill=colorPrimario!70] coordinates {
    (Rojos,5)
    (Azules,3)
    (Blancos,7)
  };
  \legend{Frecuencia}
  \end{axis}
\end{tikzpicture}
\end{center}
\end{frame}

\begin{frame}[fragile]{Frecuencia Relativa}
\begin{itemize}
  \item Es la proporción del total
\end{itemize}

\begin{itemize}
  \item Responde: ¿qué parte del todo representa?
\end{itemize}

\begin{center}
\colorbox{colorFondo}{%
  \parbox{0.8\textwidth}{%
    \begin{center}
    \Large\color{colorPrimario}
    $\text{Frec. Relativa} = \frac{\text{Frec. Absoluta}}{\text{Total}}$
    \end{center}
  }%
}
\end{center}

\begin{alertblock}{💡 Nota importante}
¡Es como saber qué porción de pizza te toca! 🍕
\end{alertblock}

\begin{exampleblock}{✨ Ejemplo}
Si hay 10 autos y 5 son rojos: 5/10 = 0.5 = 50%
\end{exampleblock}

\end{frame}

\begin{frame}[fragile]{Ejemplo Paso a Paso}
\begin{exampleblock}{🎯 Problema}
Notas de 6 estudiantes: 15, 18, 15, 20, 18, 15
\end{exampleblock}

\begin{itemize}
  \item Paso 1: Identificar valores diferentes: 15, 18, 20
\end{itemize}

\begin{itemize}
  \item Paso 2: Contar cada uno (Frec. Absoluta)
\end{itemize}

\begin{itemize}
  \item Paso 3: Calcular proporción (Frec. Relativa)
\end{itemize}

\begin{alertblock}{💡 Nota importante}
¡Vamos a construir la tabla juntos!
\end{alertblock}

\end{frame}

\begin{frame}[fragile]{Tabla Completa del Ejemplo}
\begin{center}
\begin{tabular}{ccc}
\toprule
\rowcolor{colorPrimario!20}
\textbf{Nota} & \textbf{Frec. Abs.} & \textbf{Frec. Rel.} \\
\midrule
\rowcolor{colorFondo}
15 & 3 & 3/6 = 0.50 \\
18 & 2 & 2/6 = 0.33 \\
\rowcolor{colorFondo}
20 & 1 & 1/6 = 0.17 \\
\bottomrule
\end{tabular}
\end{center}

\begin{itemize}
  \item Total de estudiantes: 6
\end{itemize}

\begin{center}
\colorbox{colorFondo}{%
  \parbox{0.9\textwidth}{%
    \begin{align*}
    ¡La nota 15 es la más frecuente! Aparece en la mitad de casos
    \end{align*}
  }%
}
\end{center}

\begin{alertblock}{💡 Nota importante}
Nota: todas las frec. relativas suman 1.00 (o 100%)
\end{alertblock}

\end{frame}

\begin{frame}[fragile]{Gráfico del Ejemplo - Notas}
\begin{center}
\begin{tikzpicture}
  \begin{axis}[
    ybar,
    width=0.8\textwidth,
    height=0.45\textheight,
    bar width=15pt,
    ylabel={Frecuencia Absoluta},
    xlabel={Calificaciones},
    symbolic x coords={Nota 15,Nota 18,Nota 20},
    xtick=data,
    x tick label style={rotate=45, anchor=east},
    ymin=0,
    enlarge x limits=0.15,
    legend style={at={(0.5,-0.25)}, anchor=north, legend columns=-1},
    nodes near coords,
    every node near coord/.append style={font=\footnotesize},
    grid=major,
    ymajorgrids=true,
    grid style={dashed,gray!30}
  ]
  \addplot[fill=colorPrimario!70] coordinates {
    (Nota 15,3)
    (Nota 18,2)
    (Nota 20,1)
  };
  \legend{Estudiantes}
  \end{axis}
\end{tikzpicture}
\end{center}
\end{frame}

\begin{frame}[fragile]{Gráfico de Barras 📊}
\begin{itemize}
  \item Barras verticales u horizontales
\end{itemize}

\begin{itemize}
  \item Cada barra = una categoría
\end{itemize}

\begin{itemize}
  \item Altura de barra = frecuencia
\end{itemize}

\begin{alertblock}{💡 Nota importante}
¡Perfecto para comparar categorías de un vistazo!
\end{alertblock}

\begin{exampleblock}{✨ Ejemplo}
Deportes favoritos: Fútbol (10), Básquet (7), Vóley (5)
\end{exampleblock}

\begin{itemize}
  \item ¡Las barras muestran claramente cuál es más popular! ⚽
\end{itemize}

\end{frame}

\begin{frame}[fragile]{Ejemplo - Deportes Favoritos}
\begin{center}
\begin{tikzpicture}
  \begin{axis}[
    ybar,
    width=0.8\textwidth,
    height=0.45\textheight,
    bar width=30pt,
    ylabel={Votos},
    xlabel={Deporte},
    symbolic x coords={Fútbol,Básquet,Vóley},
    xtick=data,
    x tick label style={rotate=45, anchor=east},
    ymin=0,
    enlarge x limits=0.15,
    legend style={at={(0.5,-0.25)}, anchor=north, legend columns=-1},
    nodes near coords,
    every node near coord/.append style={font=\footnotesize},
    grid=major,
    ymajorgrids=true,
    grid style={dashed,gray!30}
  ]
  \addplot[fill=colorPrimario!70] coordinates {
    (Fútbol,10)
    (Básquet,7)
    (Vóley,5)
  };
  \legend{Preferencias}
  \end{axis}
\end{tikzpicture}
\end{center}
\end{frame}

\begin{frame}[fragile]{Histograma 📈}
\begin{itemize}
  \item Similar a barras, pero para datos numéricos continuos
\end{itemize}

\begin{itemize}
  \item Agrupa datos en rangos o intervalos
\end{itemize}

\begin{itemize}
  \item Las barras están pegadas (sin espacios)
\end{itemize}

\begin{exampleblock}{✨ Ejemplo}
Edades: 10-15, 15-20, 20-25 años
\end{exampleblock}

\begin{alertblock}{💡 Nota importante}
¡Muestra cómo se distribuyen los datos!
\end{alertblock}

\end{frame}

\begin{frame}[fragile]{Gráfico Circular (Pastel) 🥧}
\begin{itemize}
  \item Un círculo dividido en porciones
\end{itemize}

\begin{itemize}
  \item Cada porción = una categoría
\end{itemize}

\begin{itemize}
  \item Tamaño de porción = frecuencia relativa
\end{itemize}

\begin{alertblock}{💡 Nota importante}
¡Ideal para ver proporciones del 100%!
\end{alertblock}

\begin{exampleblock}{✨ Ejemplo}
Presupuesto familiar: 50% comida, 30% vivienda, 20% otros
\end{exampleblock}

\end{frame}

\begin{frame}[fragile]{Ejemplo - Presupuesto Familiar}
\begin{center}
\begin{tikzpicture}
  \pie[
    radius=3,
    text=legend,
    color={colorPrimario!70, colorAccento!70, colorSecundario!70, colorAdvertencia!70, colorMorado!70, colorNaranja!70}
  ]{
    50/Comida,    30/Vivienda,    20/Otros  }
\end{tikzpicture}
\end{center}
\end{frame}

\begin{frame}[fragile]{Gráfico de Líneas 📉}
\begin{itemize}
  \item Puntos conectados con líneas
\end{itemize}

\begin{itemize}
  \item Perfecto para mostrar cambios en el tiempo
\end{itemize}

\begin{exampleblock}{✨ Ejemplo}
Temperatura de la semana durante 7 días
\end{exampleblock}

\begin{alertblock}{💡 Nota importante}
¡Puedes ver tendencias: ¿sube? ¿baja? ¿se mantiene?
\end{alertblock}

\begin{itemize}
  \item Las líneas te cuentan la historia del cambio 📖
\end{itemize}

\end{frame}

\begin{frame}[fragile]{Ejemplo - Temperatura Semanal}
\begin{center}
\begin{tikzpicture}
  \begin{axis}[
    width=0.85\textwidth,
    height=0.6\textheight,
    ylabel={Temperatura (°C)},
    xlabel={Día},
    grid=both,
    major grid style={line width=.2pt,draw=gray!50},
    minor grid style={line width=.1pt,draw=gray!20},
    legend pos=north east,
    mark size=3pt,
    line width=2pt
  ]
  \addplot[color=colorPrimario, mark=*] coordinates {
    (1,22)
    (2,24)
    (3,21)
    (4,23)
    (5,25)
    (6,20)
    (7,22)
  };
  \legend{Temperatura}
  \end{axis}
\end{tikzpicture}
\end{center}
\end{frame}

\begin{frame}[fragile]{¿Cuál gráfico usar?}
\begin{center}
\begin{tabular}{cc}
\toprule
\rowcolor{colorPrimario!20}
\textbf{Tipo de datos} & \textbf{Gráfico ideal} \\
\midrule
\rowcolor{colorFondo}
Categorías & Barras o Circular \\
Números continuos & Histograma \\
\rowcolor{colorFondo}
Cambios en tiempo & Líneas \\
\bottomrule
\end{tabular}
\end{center}

\begin{alertblock}{💡 Nota importante}
¡Elige según lo que quieras mostrar! 🎯
\end{alertblock}

\end{frame}

\begin{frame}[fragile]{El Promedio - ¿Qué es?}
\begin{itemize}
  \item También llamado 'media aritmética'
\end{itemize}

\begin{itemize}
  \item Es el valor central, el punto de balance ⚖️
\end{itemize}

\begin{alertblock}{💡 Nota importante}
¡Imagina repartir todo en partes iguales!
\end{alertblock}

\begin{exampleblock}{✨ Ejemplo}
Si 3 amigos tienen 3, 6 y 9 caramelos, el promedio es repartir equitativamente
\end{exampleblock}

\end{frame}

\begin{frame}[fragile]{¿Cómo calcular el Promedio?}
\begin{itemize}
  \item Paso 1: Suma todos los valores
\end{itemize}

\begin{itemize}
  \item Paso 2: Divide entre cuántos valores hay
\end{itemize}

\begin{center}
\colorbox{colorFondo}{%
  \parbox{0.8\textwidth}{%
    \begin{center}
    \Large\color{colorPrimario}
    $\bar{x} = \frac{\text{suma de todos}}{\text{cantidad de datos}}$
    \end{center}
  }%
}
\end{center}

\begin{alertblock}{💡 Nota importante}
¡Es como compartir todo y ver cuánto le toca a cada uno! 🍰
\end{alertblock}

\end{frame}

\begin{frame}[fragile]{Ejemplo del Promedio}
\begin{exampleblock}{✨ Ejemplo}
Datos: 5, 8, 10, 12, 15
\end{exampleblock}

\begin{itemize}
  \item Paso 1: Sumar todos
\end{itemize}

\begin{center}
\colorbox{colorFondo}{%
  \parbox{0.9\textwidth}{%
    \begin{align*}
    5 + 8 + 10 + 12 + 15 = 50
    \end{align*}
  }%
}
\end{center}

\begin{itemize}
  \item Paso 2: Dividir entre cantidad (5 datos)
\end{itemize}

\begin{center}
\colorbox{colorFondo}{%
  \parbox{0.9\textwidth}{%
    \begin{align*}
    \bar{x} = \frac{50}{5} = 10
    \end{align*}
  }%
}
\end{center}

\begin{alertblock}{💡 Nota importante}
¡El promedio es 10! Es el valor 'típico' del grupo
\end{alertblock}

\end{frame}

\begin{frame}[fragile]{Practicando con Frutas 🍎}
\begin{exampleblock}{🎯 Problema}
Manzanas vendidas por día: 12, 15, 10, 18, 20
\end{exampleblock}

\begin{itemize}
  \item ¿Cuántas manzanas se venden en promedio por día?
\end{itemize}

\end{frame}

\begin{frame}[fragile]{Visualización - Ventas de Manzanas}
\begin{center}
\begin{tikzpicture}
  \begin{axis}[
    ybar,
    width=0.8\textwidth,
    height=0.45\textheight,
    bar width=15pt,
    ylabel={Manzanas vendidas},
    xlabel={Día de la semana},
    symbolic x coords={Día 1,Día 2,Día 3,Día 4,Día 5},
    xtick=data,
    x tick label style={rotate=45, anchor=east},
    ymin=0,
    enlarge x limits=0.15,
    legend style={at={(0.5,-0.25)}, anchor=north, legend columns=-1},
    nodes near coords,
    every node near coord/.append style={font=\footnotesize},
    grid=major,
    ymajorgrids=true,
    grid style={dashed,gray!30}
  ]
  \addplot[fill=colorPrimario!70] coordinates {
    (Día 1,12)
    (Día 2,15)
    (Día 3,10)
    (Día 4,18)
    (Día 5,20)
  };
  \legend{Ventas}
  \end{axis}
\end{tikzpicture}
\end{center}
\end{frame}

\begin{frame}[fragile]{Solución - Promedio de Manzanas}
\begin{center}
\colorbox{colorFondo}{%
  \parbox{0.9\textwidth}{%
    \begin{align*}
    Suma: 12 + 15 + 10 + 18 + 20 = 75
    \end{align*}
  }%
}
\end{center}

\begin{center}
\colorbox{colorFondo}{%
  \parbox{0.9\textwidth}{%
    \begin{align*}
    Promedio: 75 ÷ 5 = 15 manzanas/día
    \end{align*}
  }%
}
\end{center}

\begin{itemize}
  \item ¡En promedio se venden 15 manzanas diarias! 🎯
\end{itemize}

\end{frame}

\begin{frame}[fragile]{Ejercicio Divertido 🌡️}
\begin{exampleblock}{🎯 Problema}
Temperaturas de la semana (°C): 22, 24, 21, 23, 25, 20, 22
\end{exampleblock}

\begin{itemize}
  \item Tarea 1: Construir tabla de frecuencias
\end{itemize}

\begin{itemize}
  \item Tarea 2: Calcular temperatura promedio
\end{itemize}

\begin{itemize}
  \item Tarea 3: ¿Qué gráfico usarías?
\end{itemize}

\begin{alertblock}{💡 Nota importante}
¡Piensa qué historia quieres contar con los datos!
\end{alertblock}

\end{frame}

\begin{frame}[fragile]{Solución - Tabla}
\begin{center}
\begin{tabular}{cc}
\toprule
\rowcolor{colorPrimario!20}
\textbf{Temperatura} & \textbf{Días (Frec. Abs.)} \\
\midrule
\rowcolor{colorFondo}
20°C & 1 \\
21°C & 1 \\
\rowcolor{colorFondo}
22°C & 2 \\
23°C & 1 \\
\rowcolor{colorFondo}
24°C & 1 \\
25°C & 1 \\
\bottomrule
\end{tabular}
\end{center}

\begin{alertblock}{💡 Nota importante}
22°C es la más frecuente (aparece 2 veces)
\end{alertblock}

\end{frame}

\begin{frame}[fragile]{Gráfico - Distribución de Temperaturas}
\begin{center}
\begin{tikzpicture}
  \begin{axis}[
    ybar,
    width=0.8\textwidth,
    height=0.45\textheight,
    bar width=15pt,
    ylabel={Frecuencia (días)},
    xlabel={Temperatura},
    symbolic x coords={20°C,21°C,22°C,23°C,24°C,25°C},
    xtick=data,
    x tick label style={rotate=45, anchor=east},
    ymin=0,
    enlarge x limits=0.15,
    legend style={at={(0.5,-0.25)}, anchor=north, legend columns=-1},
    nodes near coords,
    every node near coord/.append style={font=\footnotesize},
    grid=major,
    ymajorgrids=true,
    grid style={dashed,gray!30}
  ]
  \addplot[fill=colorPrimario!70] coordinates {
    (20°C,1)
    (21°C,1)
    (22°C,2)
    (23°C,1)
    (24°C,1)
    (25°C,1)
  };
  \legend{Distribución}
  \end{axis}
\end{tikzpicture}
\end{center}
\end{frame}

\begin{frame}[fragile]{Solución - Promedio}
\begin{center}
\colorbox{colorFondo}{%
  \parbox{0.9\textwidth}{%
    \begin{align*}
    Suma: 22+24+21+23+25+20+22 = 157
    \end{align*}
  }%
}
\end{center}

\begin{center}
\colorbox{colorFondo}{%
  \parbox{0.9\textwidth}{%
    \begin{align*}
    Promedio: 157 ÷ 7 = 22.43°C
    \end{align*}
  }%
}
\end{center}

\begin{itemize}
  \item Interpretación:
\end{itemize}

\begin{itemize}
  \item La temperatura típica de la semana fue 22.4°C
\end{itemize}

\begin{alertblock}{💡 Nota importante}
¡Es una semana cálida y estable! ☀️
\end{alertblock}

\end{frame}

\begin{frame}[fragile]{Solución - Gráfico}
\begin{itemize}
  \item Mejor opción: Gráfico de líneas 📈
\end{itemize}

\begin{itemize}
  \item ¿Por qué?
\end{itemize}

\begin{itemize}
  \item ✓ Muestra cambios día a día
\end{itemize}

\begin{itemize}
  \item ✓ Podemos ver si subió o bajó
\end{itemize}

\begin{itemize}
  \item ✓ Identifica tendencias
\end{itemize}

\begin{alertblock}{💡 Nota importante}
También podríamos usar barras para comparar cada día
\end{alertblock}

\end{frame}

\begin{frame}[fragile]{¡Resumen de la clase!}
\begin{itemize}
  \item ✅ Tablas: organizan datos claramente
\end{itemize}

\begin{itemize}
  \item ✅ Frecuencia absoluta: conteo directo
\end{itemize}

\begin{itemize}
  \item ✅ Frecuencia relativa: proporción del total
\end{itemize}

\begin{itemize}
  \item ✅ Gráficos: visualizan datos
\end{itemize}

\begin{itemize}
  \item ✅ Promedio: valor típico del grupo
\end{itemize}

\begin{alertblock}{💡 Nota importante}
¡Ahora puedes organizar y entender cualquier conjunto de datos! 🎉
\end{alertblock}

\end{frame}


% Diapositiva final atractiva
\begin{frame}[plain]
\begin{tikzpicture}[remember picture,overlay]
  \fill[colorAccento!20] (current page.south west) rectangle (current page.north east);
  \node[font=\Huge\bfseries,text=colorPrimario] at (current page.center) {¿Preguntas? 🤔};
  \node[font=\large,text=colorSecundario,below=1.5cm] at (current page.center) {¡Sigue aprendiendo! 🚀};
\end{tikzpicture}
\end{frame}

\end{document}