\documentclass[aspectratio=169]{beamer}

% Tema moderno y lúdico
\usetheme{Boadilla}
\usecolortheme{dolphin}

% Paquetes esenciales
\usepackage[utf8]{inputenc}
\usepackage[spanish]{babel}
\usepackage{amsmath}
\usepackage{amssymb}
\usepackage{tikz}
\usetikzlibrary{shapes, arrows, positioning, calc}
\usepackage{pgfplots}
\pgfplotsset{compat=1.18}
\usepackage{pgf-pie}
\usepackage{booktabs}
\usepackage{xcolor}
\usepackage{colortbl}
\usepackage{newunicodechar}

% Configurar emojis como texto simple
\newunicodechar{🎲}{\textsf{[dado]}}
\newunicodechar{🎯}{\textsf{[objetivo]}}
\newunicodechar{📋}{\textsf{[lista]}}
\newunicodechar{⭐}{\textsf{[estrella]}}
\newunicodechar{🔮}{\textsf{[bola]}}
\newunicodechar{📏}{\textsf{[regla]}}
\newunicodechar{🃏}{\textsf{[carta]}}
\newunicodechar{🔄}{\textsf{[ciclo]}}
\newunicodechar{🔢}{\textsf{[numeros]}}
\newunicodechar{➕}{\textsf{[+]}}
\newunicodechar{✖}{\textsf{[x]}}
\newunicodechar{🔗}{\textsf{[cadena]}}
\newunicodechar{🪙}{\textsf{[moneda]}}
\newunicodechar{🔑}{\textsf{[llave]}}
\newunicodechar{🔍}{\textsf{[lupa]}}
\newunicodechar{📊}{\textsf{[grafico]}}
\newunicodechar{🎴}{\textsf{[cartas]}}
\newunicodechar{💪}{\textsf{[fuerza]}}
\newunicodechar{🌟}{\textsf{[estrella]}}
\newunicodechar{💡}{\textsf{[idea]}}
\newunicodechar{✨}{\textsf{[brillos]}}
\newunicodechar{✅}{\textsf{[check]}}
\newunicodechar{🎓}{\textsf{[gorro]}}
\newunicodechar{🤔}{\textsf{[pensar]}}
\newunicodechar{🚀}{\textsf{[cohete]}}
\newunicodechar{♥}{\ensuremath{\heartsuit}}
\newunicodechar{₁}{\textsubscript{1}}
\newunicodechar{₂}{\textsubscript{2}}
\newunicodechar{✖}{\textsf{x}}
\newunicodechar{✓}{\textsf{[ok]}}
\newunicodechar{⚫}{\textsf{[punto]}}
\newunicodechar{👥}{\textsf{[grupo]}}
\newunicodechar{️}{}

% Colores personalizados vibrantes y lúdicos
\definecolor{colorPrimario}{RGB}{41, 128, 185}      % Azul vibrante
\definecolor{colorSecundario}{RGB}{231, 76, 60}     % Rojo coral
\definecolor{colorAccento}{RGB}{46, 204, 113}       % Verde esmeralda
\definecolor{colorAdvertencia}{RGB}{241, 196, 15}   % Amarillo dorado
\definecolor{colorMorado}{RGB}{155, 89, 182}        % Morado amigable
\definecolor{colorNaranja}{RGB}{230, 126, 34}       % Naranja cálido
\definecolor{colorFondo}{RGB}{236, 240, 241}        % Gris claro de fondo

% Configuración del tema
\setbeamercolor{structure}{fg=colorPrimario}
\setbeamercolor{palette primary}{bg=colorPrimario,fg=white}
\setbeamercolor{palette secondary}{bg=colorSecundario,fg=white}
\setbeamercolor{palette tertiary}{bg=colorAccento,fg=white}
\setbeamercolor{block title}{bg=colorPrimario!90,fg=white}
\setbeamercolor{block body}{bg=colorPrimario!10,fg=black}
\setbeamercolor{block title example}{bg=colorAccento!90,fg=white}
\setbeamercolor{block body example}{bg=colorAccento!10,fg=black}
\setbeamercolor{block title alerted}{bg=colorSecundario!90,fg=white}
\setbeamercolor{block body alerted}{bg=colorSecundario!10,fg=black}

% Plantillas personalizadas
\setbeamertemplate{navigation symbols}{}
\setbeamertemplate{footline}[frame number]
\setbeamertemplate{itemize items}[circle]
\setbeamertemplate{enumerate items}[circle]
\setbeamerfont{title}{size=\huge,series=\bfseries}
\setbeamerfont{frametitle}{size=\Large,series=\bfseries}

% Comandos personalizados para énfasis
\renewcommand{\alert}[1]{\textcolor{colorSecundario}{\textbf{#1}}}
\newcommand{\highlight}[1]{\textcolor{colorPrimario}{\textbf{#1}}}
\newcommand{\importante}[1]{\textcolor{colorAdvertencia}{\textbf{#1}}}
\newcommand{\exito}[1]{\textcolor{colorAccento}{\textbf{#1}}}

% Variables del documento (serán reemplazadas por Jinja2)
\title{Medidas de Posición}
\subtitle{¡Descubriendo dónde está cada dato en el grupo! }
\author{Probabilidad y Estadística}
\institute{Aprendiendo con diversión 🎓}
\date{\today}

\begin{document}

% Página de título con diseño atractivo
\begin{frame}[plain]
\begin{tikzpicture}[remember picture,overlay]
  % Fondo decorativo
  \fill[colorPrimario!20] (current page.south west) rectangle (current page.north east);
  \fill[colorAccento!30] (current page.south west) -- (current page.south east) -- 
        ([yshift=-3cm]current page.north east) -- ([yshift=-3cm]current page.north west) -- cycle;
\end{tikzpicture}
\titlepage
\end{frame}

% Diapositivas generadas dinámicamente
\begin{frame}[fragile]{¿Qué son las Medidas de Posición?}
\begin{alertblock}{💡 Nota importante}
¡Imagina una fila de personas ordenadas por altura! 
\end{alertblock}

\begin{itemize}
  \item Nos dicen en qué lugar está un dato comparado con los demás.
\end{itemize}

\begin{itemize}
  \item Responden preguntas como:
\end{itemize}

\begin{itemize}
  \item ¿Quién está justo en el medio de la fila?
\end{itemize}

\begin{itemize}
  \item ¿Qué altura tiene la persona que es más alta que el 75 por ciento del grupo?
\end{itemize}

\begin{exampleblock}{✨ Ejemplo}
¡Es como saber en qué puesto quedaste en una carrera! 
\end{exampleblock}

\end{frame}

\begin{frame}[fragile]{La Mediana: ¡El Corazón de los Datos!}
\begin{itemize}
  \item Es el dato que está JUSTO en el medio de la lista ordenada.
\end{itemize}

\begin{itemize}
  \item La mitad (50 por ciento) de los datos son más pequeños que ella.
\end{itemize}

\begin{itemize}
  \item La otra mitad (50 por ciento) son más grandes.
\end{itemize}

\begin{alertblock}{💡 Nota importante}
¡Es el estudiante del centro en una fila ordenada por notas!
\end{alertblock}

\begin{exampleblock}{✨ Ejemplo}
Alturas: 1.50m, 1.60m, 1.70m, 1.80m, 1.90m
\end{exampleblock}

\begin{center}
\colorbox{colorFondo}{%
  \parbox{0.9\textwidth}{%
    \begin{align*}
    La Mediana es 1.70m (¡el del medio!)
    \end{align*}
  }%
}
\end{center}

\end{frame}

\begin{frame}[fragile]{Visualización - La Mediana en el Centro}
\begin{center}
\begin{tikzpicture}
  \begin{axis}[
    ybar,
    width=0.8\textwidth,
    height=0.45\textheight,
    bar width=15pt,
    ylabel={Personas},
    xlabel={Altura},
    symbolic x coords={1.50m,1.60m,1.70m (Mediana),1.80m,1.90m},
    xtick=data,
    x tick label style={rotate=45, anchor=east},
    ymin=0,
    enlarge x limits=0.15,
    legend style={at={(0.5,-0.25)}, anchor=north, legend columns=-1},
    nodes near coords,
    every node near coord/.append style={font=\footnotesize},
    grid=major,
    ymajorgrids=true,
    grid style={dashed,gray!30}
  ]
  \addplot[fill=colorPrimario!70] coordinates {
    (1.50m,1)
    (1.60m,1)
    (1.70m (Mediana),1)
    (1.80m,1)
    (1.90m,1)
  };
  \legend{Datos ordenados}
  \end{axis}
\end{tikzpicture}
\end{center}
\end{frame}

\begin{frame}[fragile]{Cómo Encontrar la Mediana}
\begin{itemize}
  \item Paso 1: ¡Ordena los datos de menor a mayor!
\end{itemize}

\begin{itemize}
  \item Paso 2: Si la cantidad de datos es IMPAR, la mediana es el del centro.
\end{itemize}

\begin{itemize}
  \item Paso 3: Si la cantidad es PAR, la mediana es el promedio de los DOS del centro.
\end{itemize}

\begin{exampleblock}{✨ Ejemplo}
Datos: 3, 7, 2, 9, 5 → Ordenados: 2, 3, 5, 7, 9
\end{exampleblock}

\begin{center}
\colorbox{colorFondo}{%
  \parbox{0.9\textwidth}{%
    \begin{align*}
    La Mediana es 5 (el valor del medio).
    \end{align*}
  }%
}
\end{center}

\end{frame}

\begin{frame}[fragile]{Mediana con Cantidad Par de Datos}
\begin{exampleblock}{✨ Ejemplo}
Datos: 4, 8, 6, 2, 10, 12
\end{exampleblock}

\begin{itemize}
  \item Paso 1: Ordenar → 2, 4, 6, 8, 10, 12
\end{itemize}

\begin{itemize}
  \item Paso 2: Los dos del centro son 6 y 8.
\end{itemize}

\begin{itemize}
  \item Paso 3: ¡Calculamos su promedio!
\end{itemize}

\begin{center}
\colorbox{colorFondo}{%
  \parbox{0.9\textwidth}{%
    \begin{align*}
    Mediana = (6 + 8) / 2 = 7
    \end{align*}
  }%
}
\end{center}

\begin{alertblock}{💡 Nota importante}
¡A veces la mediana no es uno de los datos originales!
\end{alertblock}

\end{frame}

\begin{frame}[fragile]{¿Qué son los Cuartiles? ¡La Pizza de Datos!}
\begin{itemize}
  \item ¡Dividen tus datos ordenados en 4 partes iguales!
\end{itemize}

\begin{alertblock}{💡 Nota importante}
¡Como cortar una pizza en 4 trozos!  Cada trozo es un 25 por ciento.
\end{alertblock}

\begin{itemize}
  \item Q1 (Cuartil 1): El 25 por ciento de los datos es menor que este valor.
\end{itemize}

\begin{itemize}
  \item Q2 (Cuartil 2): ¡Es la Mediana! El 50 por ciento de los datos es menor.
\end{itemize}

\begin{itemize}
  \item Q3 (Cuartil 3): El 75 por ciento de los datos es menor que este valor.
\end{itemize}

\end{frame}

\begin{frame}[fragile]{Los Cuartiles como Puntos de Control}
\begin{itemize}
  \item Q1: Supera al 25 por ciento de los datos.
\end{itemize}

\begin{itemize}
  \item Q2 (Mediana): Supera al 50 por ciento de los datos.
\end{itemize}

\begin{itemize}
  \item Q3: Supera al 75 por ciento de los datos.
\end{itemize}

\begin{alertblock}{💡 Nota importante}
¡Son como las marcas de 1/4, 1/2 y 3/4 en una carrera! 
\end{alertblock}

\begin{exampleblock}{✨ Ejemplo}
Si el Q3 de salarios es S/3000, significa que el 75 por ciento de la gente gana menos que eso.
\end{exampleblock}

\end{frame}

\begin{frame}[fragile]{Ejemplo de Cuartiles Paso a Paso}
\begin{exampleblock}{🎯 Problema}
Datos ordenados: 2, 4, 6, 8, 10, 12, 14, 16, 18
\end{exampleblock}

\begin{itemize}
  \item 1. Encontramos la mediana (Q2): el valor central es 10.
\end{itemize}

\begin{center}
\colorbox{colorFondo}{%
  \parbox{0.9\textwidth}{%
    \begin{align*}
    Q2 (Mediana) = 10
    \end{align*}
  }%
}
\end{center}

\begin{itemize}
  \item 2. Ahora miramos la mitad de abajo: 2, 4, 6, 8. La mediana de este grupo es Q1.
\end{itemize}

\begin{center}
\colorbox{colorFondo}{%
  \parbox{0.9\textwidth}{%
    \begin{align*}
    Q1 = (4+6)/2 = 5
    \end{align*}
  }%
}
\end{center}

\begin{itemize}
  \item 3. Hacemos lo mismo con la mitad de arriba: 12, 14, 16, 18. La mediana es Q3.
\end{itemize}

\begin{center}
\colorbox{colorFondo}{%
  \parbox{0.9\textwidth}{%
    \begin{align*}
    Q3 = (14+16)/2 = 15
    \end{align*}
  }%
}
\end{center}

\end{frame}

\begin{frame}[fragile]{Visualización - Los Cuartiles en Acción}
\begin{center}
\begin{tikzpicture}
  \begin{axis}[
    ybar,
    width=0.8\textwidth,
    height=0.45\textheight,
    bar width=15pt,
    ylabel={Frecuencia},
    xlabel={Valores},
    symbolic x coords={2,4 (Q1=5),6,8,10 (Q2),12,14 (Q3=15),16,18},
    xtick=data,
    x tick label style={rotate=45, anchor=east},
    ymin=0,
    enlarge x limits=0.15,
    legend style={at={(0.5,-0.25)}, anchor=north, legend columns=-1},
    nodes near coords,
    every node near coord/.append style={font=\footnotesize},
    grid=major,
    ymajorgrids=true,
    grid style={dashed,gray!30}
  ]
  \addplot[fill=colorPrimario!70] coordinates {
    (2,1)
    (4 (Q1=5),1)
    (6,1)
    (8,1)
    (10 (Q2),1)
    (12,1)
    (14 (Q3=15),1)
    (16,1)
    (18,1)
  };
  \legend{Distribución}
  \end{axis}
\end{tikzpicture}
\end{center}
\end{frame}

\begin{frame}[fragile]{Interpretando los Cuartiles del Ejemplo}
\begin{exampleblock}{✨ Ejemplo}
Q1=5, Q2=10, Q3=15
\end{exampleblock}

\begin{itemize}
  \item Interpretación:
\end{itemize}

\begin{itemize}
  \item  El 25 por ciento de los datos son 5 o menos.
\end{itemize}

\begin{itemize}
  \item  El 50 por ciento de los datos son 10 o menos.
\end{itemize}

\begin{itemize}
  \item  El 75 por ciento de los datos son 15 o menos.
\end{itemize}

\begin{alertblock}{💡 Nota importante}
¡Los cuartiles nos cuentan cómo se reparten los datos!
\end{alertblock}

\end{frame}

\begin{frame}[fragile]{¿Y los Percentiles? ¡Aún más preciso!}
\begin{itemize}
  \item ¡Dividen los datos en 100 partes iguales!
\end{itemize}

\begin{alertblock}{💡 Nota importante}
¡Como una regla de 100 centímetros!  Cada centímetro es un percentil.
\end{alertblock}

\begin{itemize}
  \item El Percentil 80 (P₈₀) significa que el 80 por ciento de los datos son menores.
\end{itemize}

\begin{exampleblock}{✨ Ejemplo}
Si estás en el P₉₀ en una prueba, ¡felicidades! Superaste al 90 por ciento de los estudiantes. 
\end{exampleblock}

\end{frame}

\begin{frame}[fragile]{Cuartiles y Percentiles son Familia}
\begin{center}
\begin{tabular}{cc}
\toprule
\rowcolor{colorPrimario!20}
\textbf{Cuartil} & \textbf{Percentil} \\
\midrule
\rowcolor{colorFondo}
Q1 (25 porciento) & P25 \\
Q2 (50 porciento) & P50 \\
\rowcolor{colorFondo}
Q3 (75 porciento) & P75 \\
\bottomrule
\end{tabular}
\end{center}

\begin{alertblock}{💡 Nota importante}
¡Los cuartiles son los percentiles más famosos!
\end{alertblock}

\end{frame}

\begin{frame}[fragile]{Ejemplo de Percentiles}
\begin{exampleblock}{✨ Ejemplo}
En un examen, tu nota fue 85 puntos.
\end{exampleblock}

\begin{itemize}
  \item Te dicen que el Percentil 90 (P₉₀) fue 80 puntos.
\end{itemize}

\begin{itemize}
  \item Interpretación:
\end{itemize}

\begin{itemize}
  \item  ¡Tu nota de 85 es excelente! Superaste al 90 por ciento de la clase.
\end{itemize}

\begin{itemize}
  \item  Solo el 10 por ciento sacó una nota más alta que 80.
\end{itemize}

\begin{alertblock}{💡 Nota importante}
¡Estás en el Top 10 por ciento del grupo! 
\end{alertblock}

\end{frame}

\begin{frame}[fragile]{El Diagrama de Cajón (Box Plot)}
\begin{itemize}
  \item ¡Un súper resumen visual de tus datos!
\end{itemize}

\begin{alertblock}{💡 Nota importante}
¡Es como una radiografía de tus datos en 5 números clave! 
\end{alertblock}

\begin{itemize}
  \item Te muestra de un solo vistazo:
\end{itemize}

\begin{itemize}
  \item  El valor Mínimo
\end{itemize}

\begin{itemize}
  \item  Q1, Q2 (la mediana) y Q3
\end{itemize}

\begin{itemize}
  \item  El valor Máximo
\end{itemize}

\end{frame}

\begin{frame}[fragile]{Las Partes del Diagrama de Cajón}
\begin{itemize}
  \item  La Caja: va desde Q1 hasta Q3. ¡Contiene el 50 por ciento central de los datos!
\end{itemize}

\begin{itemize}
  \item  El ancho de la caja se llama Rango Intercuartílico (RIC).
\end{itemize}

\begin{itemize}
  \item  La línea dentro de la caja: ¡es la mediana (Q2)!
\end{itemize}

\begin{itemize}
  \item  Los Bigotes: son las líneas que van desde la caja hasta el mínimo y el máximo.
\end{itemize}

\begin{alertblock}{💡 Nota importante}
¡Un dibujo vale más que mil números!
\end{alertblock}

\end{frame}

\begin{frame}[fragile]{Dibujando un Box Plot}
\begin{exampleblock}{🎯 Problema}
Datos: 5, 7, 8, 9, 10, 12, 15, 18, 20
\end{exampleblock}

\begin{itemize}
  \item Calculamos los 5 números clave:
\end{itemize}

\begin{itemize}
  \item Mínimo=5, Q1=8, Mediana(Q2)=10, Q3=15, Máximo=20
\end{itemize}

\begin{itemize}
  \item Y dibujamos:
\end{itemize}

\begin{itemize}
  \item Bigote izquierdo: de 5 a 8.
\end{itemize}

\begin{itemize}
  \item Caja: de 8 a 15 (con una línea en 10).
\end{itemize}

\begin{itemize}
  \item Bigote derecho: de 15 a 20.
\end{itemize}

\end{frame}

\begin{frame}[fragile]{Valores Atípicos (Outliers): ¡Los Rebeldes!}
\begin{itemize}
  \item Son datos que están MUY lejos del resto del grupo.
\end{itemize}

\begin{alertblock}{💡 Nota importante}
¡Como un pingüino en el desierto! No encaja. 
\end{alertblock}

\begin{itemize}
  \item ¿Cómo los encontramos? Con una regla simple:
\end{itemize}

\begin{itemize}
  \item Un dato es atípico si está 'demasiado lejos' de la caja.
\end{itemize}

\begin{itemize}
  \item Demasiado lejos = más de 1.5 veces el ancho de la caja (RIC).
\end{itemize}

\end{frame}

\begin{frame}[fragile]{Ejemplo de Valor Atípico}
\begin{exampleblock}{✨ Ejemplo}
Edades en un grupo: 10, 12, 11, 13, 12, y... ¡50!
\end{exampleblock}

\begin{itemize}
  \item La mayoría de edades están entre 10 y 13.
\end{itemize}

\begin{itemize}
  \item El valor 50 está súper alejado del resto.
\end{itemize}

\begin{alertblock}{💡 Nota importante}
¡50 es un valor atípico! En el gráfico, se marca con un punto especial  para no distorsionar el dibujo.
\end{alertblock}

\end{frame}

\begin{frame}[fragile]{Aplicación Práctica - Salarios de una Empresa}
\begin{exampleblock}{🎯 Problema}
Salarios mensuales (S/.): 1200, 1500, 1800, 2000, 2200, 2500, 3000, 3500, 5000
\end{exampleblock}

\begin{itemize}
  \item Misión: Analizar estos salarios con nuestras nuevas herramientas.
\end{itemize}

\begin{itemize}
  \item Paso 1: Calcular cuartiles.
\end{itemize}

\begin{itemize}
  \item Paso 2: Dibujar el diagrama de cajón.
\end{itemize}

\begin{itemize}
  \item Paso 3: Buscar salarios 'rebeldes' (atípicos).
\end{itemize}

\end{frame}

\begin{frame}[fragile]{Solución - Cuartiles de Salarios}
\begin{itemize}
  \item Mínimo = 1200, Máximo = 5000
\end{itemize}

\begin{itemize}
  \item Mediana (Q2) = 2200 (el salario del medio)
\end{itemize}

\begin{itemize}
  \item Q1 (mediana de la mitad de abajo) = 1650
\end{itemize}

\begin{itemize}
  \item Q3 (mediana de la mitad de arriba) = 3250
\end{itemize}

\end{frame}

\begin{frame}[fragile]{Solución - ¿Hay Salarios Atípicos?}
\begin{itemize}
  \item El ancho de la caja (RIC) es 3250 - 1650 = 1600.
\end{itemize}

\begin{itemize}
  \item Calculamos los 'límites de lo normal':
\end{itemize}

\begin{itemize}
  \item Límite inferior: 1650 - 1.5 * 1600 = -750 (imposible un salario negativo)
\end{itemize}

\begin{itemize}
  \item Límite superior: 3250 + 1.5 * 1600 = 5650
\end{itemize}

\begin{itemize}
  \item Todos los salarios están dentro de estos límites.
\end{itemize}

\begin{alertblock}{💡 Nota importante}
¡No hay valores atípicos! El salario de 5000, aunque alto, no se considera 'rebelde'.
\end{alertblock}

\end{frame}

\begin{frame}[fragile]{Interpretación de los Salarios}
\begin{itemize}
  \item  El 50 por ciento central de los trabajadores gana entre S/1650 y S/3250.
\end{itemize}

\begin{itemize}
  \item  El salario más representativo (mediana) es S/2200.
\end{itemize}

\begin{itemize}
  \item  El 25 por ciento gana menos de S/1650 (los más nuevos quizás).
\end{itemize}

\begin{itemize}
  \item  El 25 por ciento gana más de S/3250 (los de más experiencia).
\end{itemize}

\begin{alertblock}{💡 Nota importante}
¡El box plot nos contó toda la historia de los salarios!
\end{alertblock}

\end{frame}

\begin{frame}[fragile]{Comparando Grupos con Box Plots}
\begin{itemize}
  \item ¡Son geniales para comparar dos o más grupos a la vez!
\end{itemize}

\begin{exampleblock}{✨ Ejemplo}
Comparar las notas de 2 salones: 5to A vs 5to B.
\end{exampleblock}

\begin{itemize}
  \item De un vistazo puedes ver:
\end{itemize}

\begin{itemize}
  \item  ¿Qué salón tiene mejores notas en general?
\end{itemize}

\begin{itemize}
  \item  ¿En qué salón las notas son más parecidas entre sí?
\end{itemize}

\begin{itemize}
  \item  ¿Hay algún alumno con una nota súper alta o baja (atípico)?
\end{itemize}

\begin{alertblock}{💡 Nota importante}
¡La comparación es súper visual y rápida! 
\end{alertblock}

\end{frame}

\begin{frame}[fragile]{Ejercicio de Práctica}
\begin{exampleblock}{🎯 Problema}
Edades de un grupo de amigos: 15, 16, 15, 17, 16, 18, 15, 20, 16, 25
\end{exampleblock}

\begin{itemize}
  \item Tu Misión:
\end{itemize}

\begin{itemize}
  \item 1. Ordena los datos.
\end{itemize}

\begin{itemize}
  \item 2. Calcula Q1, Q2 (mediana) y Q3.
\end{itemize}

\begin{itemize}
  \item 3. ¿Es la edad de 25 un valor atípico en este grupo?
\end{itemize}

\begin{alertblock}{💡 Nota importante}
¡Intenta resolverlo antes de ver la solución!
\end{alertblock}

\end{frame}

\begin{frame}[fragile]{Solución del Ejercicio}
\begin{itemize}
  \item Ordenado: 15, 15, 15, 16, 16, 16, 17, 18, 20, 25
\end{itemize}

\begin{itemize}
  \item Q1 = 15 (la mediana de la primera mitad)
\end{itemize}

\begin{itemize}
  \item Q2 = 16 (el promedio de 16 y 16)
\end{itemize}

\begin{itemize}
  \item Q3 = 18 (la mediana de la segunda mitad)
\end{itemize}

\begin{itemize}
  \item Límite superior para atípicos: 18 + 1.5 * (18 - 15) = 22.5
\end{itemize}

\begin{itemize}
  \item Como 25 es mayor que 22.5...
\end{itemize}

\begin{alertblock}{💡 Nota importante}
¡Sí! 25 es un valor atípico. Probablemente es un amigo mayor que el resto del grupo. 
\end{alertblock}

\end{frame}

\begin{frame}[fragile]{¡Resumen de la Misión!}
\begin{itemize}
  \item  Mediana: el dato del corazón del grupo.
\end{itemize}

\begin{itemize}
  \item  Cuartiles: dividen los datos en 4 trozos de pizza.
\end{itemize}

\begin{itemize}
  \item  Percentiles: dividen los datos en 100 mini-partes.
\end{itemize}

\begin{itemize}
  \item  Box Plot: la radiografía de tus datos.
\end{itemize}

\begin{itemize}
  \item  Valores Atípicos: los 'rebeldes' del grupo.
\end{itemize}

\begin{alertblock}{💡 Nota importante}
¡Ahora puedes analizar cualquier grupo y saber dónde está cada quien! 
\end{alertblock}

\end{frame}


% Diapositiva final atractiva
\begin{frame}[plain]
\begin{tikzpicture}[remember picture,overlay]
  \fill[colorAccento!20] (current page.south west) rectangle (current page.north east);
  \node[font=\Huge\bfseries,text=colorPrimario] at (current page.center) {¿Preguntas? 🤔};
  \node[font=\large,text=colorSecundario,below=1.5cm] at (current page.center) {¡Sigue aprendiendo! 🚀};
\end{tikzpicture}
\end{frame}

\end{document}