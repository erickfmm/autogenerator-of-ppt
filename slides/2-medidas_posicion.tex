\documentclass[aspectratio=169]{beamer}

% Tema moderno y lúdico
\usetheme{Boadilla}
\usecolortheme{dolphin}

% Paquetes esenciales
\usepackage[utf8]{inputenc}
\usepackage[spanish]{babel}
\usepackage{amsmath}
\usepackage{amssymb}
\usepackage{tikz}
\usetikzlibrary{shapes, arrows, positioning, calc}
\usepackage{pgfplots}
\pgfplotsset{compat=1.18}
\usepackage{pgf-pie}
\usepackage{booktabs}
\usepackage{xcolor}
\usepackage{colortbl}
\usepackage{newunicodechar}

% Configurar emojis como texto simple
\newunicodechar{🎲}{\textsf{[dado]}}
\newunicodechar{🎯}{\textsf{[objetivo]}}
\newunicodechar{📋}{\textsf{[lista]}}
\newunicodechar{⭐}{\textsf{[estrella]}}
\newunicodechar{🔮}{\textsf{[bola]}}
\newunicodechar{📏}{\textsf{[regla]}}
\newunicodechar{🃏}{\textsf{[carta]}}
\newunicodechar{🔄}{\textsf{[ciclo]}}
\newunicodechar{🔢}{\textsf{[numeros]}}
\newunicodechar{➕}{\textsf{[+]}}
\newunicodechar{✖}{\textsf{[x]}}
\newunicodechar{🔗}{\textsf{[cadena]}}
\newunicodechar{🪙}{\textsf{[moneda]}}
\newunicodechar{🔑}{\textsf{[llave]}}
\newunicodechar{🔍}{\textsf{[lupa]}}
\newunicodechar{📊}{\textsf{[grafico]}}
\newunicodechar{🎴}{\textsf{[cartas]}}
\newunicodechar{💪}{\textsf{[fuerza]}}
\newunicodechar{🌟}{\textsf{[estrella]}}
\newunicodechar{💡}{\textsf{[idea]}}
\newunicodechar{✨}{\textsf{[brillos]}}
\newunicodechar{✅}{\textsf{[check]}}
\newunicodechar{🎓}{\textsf{[gorro]}}
\newunicodechar{🤔}{\textsf{[pensar]}}
\newunicodechar{🚀}{\textsf{[cohete]}}
\newunicodechar{♥}{\ensuremath{\heartsuit}}
\newunicodechar{₁}{\textsubscript{1}}
\newunicodechar{₂}{\textsubscript{2}}
\newunicodechar{✖}{\textsf{x}}
\newunicodechar{✓}{\textsf{[ok]}}
\newunicodechar{⚫}{\textsf{[punto]}}
\newunicodechar{👥}{\textsf{[grupo]}}
\newunicodechar{️}{}

% Colores personalizados vibrantes y lúdicos
\definecolor{colorPrimario}{RGB}{41, 128, 185}      % Azul vibrante
\definecolor{colorSecundario}{RGB}{231, 76, 60}     % Rojo coral
\definecolor{colorAccento}{RGB}{46, 204, 113}       % Verde esmeralda
\definecolor{colorAdvertencia}{RGB}{241, 196, 15}   % Amarillo dorado
\definecolor{colorMorado}{RGB}{155, 89, 182}        % Morado amigable
\definecolor{colorNaranja}{RGB}{230, 126, 34}       % Naranja cálido
\definecolor{colorFondo}{RGB}{236, 240, 241}        % Gris claro de fondo

% Configuración del tema
\setbeamercolor{structure}{fg=colorPrimario}
\setbeamercolor{palette primary}{bg=colorPrimario,fg=white}
\setbeamercolor{palette secondary}{bg=colorSecundario,fg=white}
\setbeamercolor{palette tertiary}{bg=colorAccento,fg=white}
\setbeamercolor{block title}{bg=colorPrimario!90,fg=white}
\setbeamercolor{block body}{bg=colorPrimario!10,fg=black}
\setbeamercolor{block title example}{bg=colorAccento!90,fg=white}
\setbeamercolor{block body example}{bg=colorAccento!10,fg=black}
\setbeamercolor{block title alerted}{bg=colorSecundario!90,fg=white}
\setbeamercolor{block body alerted}{bg=colorSecundario!10,fg=black}

% Plantillas personalizadas
\setbeamertemplate{navigation symbols}{}
\setbeamertemplate{footline}[frame number]
\setbeamertemplate{itemize items}[circle]
\setbeamertemplate{enumerate items}[circle]
\setbeamerfont{title}{size=\huge,series=\bfseries}
\setbeamerfont{frametitle}{size=\Large,series=\bfseries}

% Comandos personalizados para énfasis
\renewcommand{\alert}[1]{\textcolor{colorSecundario}{\textbf{#1}}}
\newcommand{\highlight}[1]{\textcolor{colorPrimario}{\textbf{#1}}}
\newcommand{\importante}[1]{\textcolor{colorAdvertencia}{\textbf{#1}}}
\newcommand{\exito}[1]{\textcolor{colorAccento}{\textbf{#1}}}

% Variables del documento (serán reemplazadas por Jinja2)
\title{Medidas de Posición}
\subtitle{¡Descubriendo dónde están los datos! 📍}
\author{Probabilidad y Estadística}
\institute{Aprendiendo con diversión 🎓}
\date{\today}

\begin{document}

% Página de título con diseño atractivo
\begin{frame}[plain]
\begin{tikzpicture}[remember picture,overlay]
  % Fondo decorativo
  \fill[colorPrimario!20] (current page.south west) rectangle (current page.north east);
  \fill[colorAccento!30] (current page.south west) -- (current page.south east) -- 
        ([yshift=-3cm]current page.north east) -- ([yshift=-3cm]current page.north west) -- cycle;
\end{tikzpicture}
\titlepage
\end{frame}

% Diapositivas generadas dinámicamente
\begin{frame}[fragile]{¿Qué son las Medidas de Posición?}
\begin{alertblock}{💡 Nota importante}
¡Imagina una fila de personas ordenadas por altura! 👥
\end{alertblock}

\begin{itemize}
  \item Nos ayudan a ubicar datos dentro de un grupo
\end{itemize}

\begin{itemize}
  \item Responden preguntas como:
\end{itemize}

\begin{itemize}
  \item ¿Dónde está el dato del medio?
\end{itemize}

\begin{itemize}
  \item ¿Cuál es el valor que supera el 75% de los datos?
\end{itemize}

\begin{exampleblock}{✨ Ejemplo}
Como saber en qué puesto quedaste en una carrera 🏃
\end{exampleblock}

\end{frame}

\begin{frame}[fragile]{La Mediana - El Valor del Centro}
\begin{itemize}
  \item Es el dato que está justo en el medio
\end{itemize}

\begin{itemize}
  \item 50% de datos están abajo, 50% arriba
\end{itemize}

\begin{alertblock}{💡 Nota importante}
¡Es como el estudiante del medio en una fila ordenada!
\end{alertblock}

\begin{exampleblock}{✨ Ejemplo}
Alturas: 1.50, 1.60, 1.70, 1.80, 1.90 m
\end{exampleblock}

\begin{center}
\colorbox{colorFondo}{%
  \parbox{0.9\textwidth}{%
    \begin{align*}
    Mediana = 1.70 m (el del medio)
    \end{align*}
  }%
}
\end{center}

\end{frame}

\begin{frame}[fragile]{Visualización - Mediana}
\begin{center}
\begin{tikzpicture}
  \begin{axis}[
    ybar,
    width=0.8\textwidth,
    height=0.45\textheight,
    bar width=15pt,
    ylabel={Frecuencia},
    xlabel={Altura},
    symbolic x coords={1.50m,1.60m,1.70m,1.80m,1.90m},
    xtick=data,
    x tick label style={rotate=45, anchor=east},
    ymin=0,
    enlarge x limits=0.15,
    legend style={at={(0.5,-0.25)}, anchor=north, legend columns=-1},
    nodes near coords,
    every node near coord/.append style={font=\footnotesize},
    grid=major,
    ymajorgrids=true,
    grid style={dashed,gray!30}
  ]
  \addplot[fill=colorPrimario!70] coordinates {
    (1.50m,1)
    (1.60m,1)
    (1.70m,1)
    (1.80m,1)
    (1.90m,1)
  };
  \legend{Datos ordenados}
  \end{axis}
\end{tikzpicture}
\end{center}
\end{frame}

\begin{frame}[fragile]{Encontrar la Mediana}
\begin{itemize}
  \item Paso 1: Ordena los datos de menor a mayor
\end{itemize}

\begin{itemize}
  \item Paso 2: Si hay cantidad impar, toma el del centro
\end{itemize}

\begin{itemize}
  \item Paso 3: Si hay cantidad par, promedia los 2 del centro
\end{itemize}

\begin{exampleblock}{✨ Ejemplo}
Datos: 3, 7, 2, 9, 5 → Ordenar: 2, 3, 5, 7, 9
\end{exampleblock}

\begin{center}
\colorbox{colorFondo}{%
  \parbox{0.9\textwidth}{%
    \begin{align*}
    Mediana = 5 (posición central)
    \end{align*}
  }%
}
\end{center}

\end{frame}

\begin{frame}[fragile]{Mediana con Cantidad Par}
\begin{exampleblock}{✨ Ejemplo}
Datos: 4, 8, 6, 2, 10, 12
\end{exampleblock}

\begin{itemize}
  \item Ordenar: 2, 4, 6, 8, 10, 12
\end{itemize}

\begin{itemize}
  \item Los dos centrales son: 6 y 8
\end{itemize}

\begin{center}
\colorbox{colorFondo}{%
  \parbox{0.9\textwidth}{%
    \begin{align*}
    Mediana = (6 + 8) ÷ 2 = 7
    \end{align*}
  }%
}
\end{center}

\begin{alertblock}{💡 Nota importante}
¡La mediana no siempre es un dato original!
\end{alertblock}

\end{frame}

\begin{frame}[fragile]{¿Qué son los Cuartiles?}
\begin{itemize}
  \item Dividen los datos en 4 partes iguales
\end{itemize}

\begin{alertblock}{💡 Nota importante}
¡Como cortar una pizza en 4 partes! 🍕
\end{alertblock}

\begin{itemize}
  \item Q1: primer cuartil (25%)
\end{itemize}

\begin{itemize}
  \item Q2: segundo cuartil = mediana (50%)
\end{itemize}

\begin{itemize}
  \item Q3: tercer cuartil (75%)
\end{itemize}

\begin{exampleblock}{✨ Ejemplo}
Separan grupos de 25%, 50%, 75% y 100%
\end{exampleblock}

\end{frame}

\begin{frame}[fragile]{Cuartiles Explicados}
\begin{itemize}
  \item Q1: 25% de datos están por debajo
\end{itemize}

\begin{itemize}
  \item Q2: 50% de datos están por debajo (mediana)
\end{itemize}

\begin{itemize}
  \item Q3: 75% de datos están por debajo
\end{itemize}

\begin{alertblock}{💡 Nota importante}
¡Son como marcas en una regla! 📏
\end{alertblock}

\begin{exampleblock}{✨ Ejemplo}
Si Q3 de salarios es 3000, significa que 75% gana menos de 3000
\end{exampleblock}

\end{frame}

\begin{frame}[fragile]{Ejemplo de Cuartiles Paso a Paso}
\begin{exampleblock}{🎯 Problema}
Datos: 2, 4, 6, 8, 10, 12, 14, 16, 18
\end{exampleblock}

\begin{itemize}
  \item Ya están ordenados (9 valores)
\end{itemize}

\begin{center}
\colorbox{colorFondo}{%
  \parbox{0.9\textwidth}{%
    \begin{align*}
    Q2 (mediana) = 10 (valor central)
    \end{align*}
  }%
}
\end{center}

\begin{itemize}
  \item Mitad inferior: 2, 4, 6, 8
\end{itemize}

\begin{center}
\colorbox{colorFondo}{%
  \parbox{0.9\textwidth}{%
    \begin{align*}
    Q1 = (4+6)÷2 = 5
    \end{align*}
  }%
}
\end{center}

\begin{itemize}
  \item Mitad superior: 12, 14, 16, 18
\end{itemize}

\begin{center}
\colorbox{colorFondo}{%
  \parbox{0.9\textwidth}{%
    \begin{align*}
    Q3 = (14+16)÷2 = 15
    \end{align*}
  }%
}
\end{center}

\end{frame}

\begin{frame}[fragile]{Visualización - Distribución de Datos}
\begin{center}
\begin{tikzpicture}
  \begin{axis}[
    ybar,
    width=0.8\textwidth,
    height=0.45\textheight,
    bar width=15pt,
    ylabel={Frecuencia},
    xlabel={Valores},
    symbolic x coords={2,4,6,8,10,12,14,16,18},
    xtick=data,
    x tick label style={rotate=45, anchor=east},
    ymin=0,
    enlarge x limits=0.15,
    legend style={at={(0.5,-0.25)}, anchor=north, legend columns=-1},
    nodes near coords,
    every node near coord/.append style={font=\footnotesize},
    grid=major,
    ymajorgrids=true,
    grid style={dashed,gray!30}
  ]
  \addplot[fill=colorPrimario!70] coordinates {
    (2,1)
    (4,1)
    (6,1)
    (8,1)
    (10,1)
    (12,1)
    (14,1)
    (16,1)
    (18,1)
  };
  \legend{Distribución}
  \end{axis}
\end{tikzpicture}
\end{center}
\end{frame}

\begin{frame}[fragile]{Interpretando los Cuartiles}
\begin{exampleblock}{✨ Ejemplo}
Q1=5, Q2=10, Q3=15
\end{exampleblock}

\begin{itemize}
  \item Interpretación:
\end{itemize}

\begin{itemize}
  \item 📊 25% de datos ≤ 5
\end{itemize}

\begin{itemize}
  \item 📊 50% de datos ≤ 10
\end{itemize}

\begin{itemize}
  \item 📊 75% de datos ≤ 15
\end{itemize}

\begin{alertblock}{💡 Nota importante}
¡Los cuartiles dividen los datos en grupos iguales!
\end{alertblock}

\end{frame}

\begin{frame}[fragile]{¿Qué son los Percentiles?}
\begin{itemize}
  \item Dividen los datos en 100 partes iguales
\end{itemize}

\begin{alertblock}{💡 Nota importante}
¡Como dividir una cinta métrica en 100 cm! 📐
\end{alertblock}

\begin{itemize}
  \item P₁, P₂, P₃, ..., P₉₉
\end{itemize}

\begin{itemize}
  \item Pₖ significa: k% de datos están por debajo
\end{itemize}

\begin{exampleblock}{✨ Ejemplo}
P₉₀ en una prueba: superaste al 90% de estudiantes 🎓
\end{exampleblock}

\end{frame}

\begin{frame}[fragile]{Relación entre Cuartiles y Percentiles}
\begin{center}
\begin{tabular}{cc}
\toprule
\rowcolor{colorPrimario!20}
\textbf{Cuartil} & \textbf{Percentil Equivalente} \\
\midrule
\rowcolor{colorFondo}
Q1 & P₂₅ \\
Q2 (mediana) & P₅₀ \\
\rowcolor{colorFondo}
Q3 & P₇₅ \\
\bottomrule
\end{tabular}
\end{center}

\begin{alertblock}{💡 Nota importante}
¡Los cuartiles son percentiles especiales!
\end{alertblock}

\begin{itemize}
  \item P₁₀ = 10% por debajo, P₉₀ = 90% por debajo
\end{itemize}

\end{frame}

\begin{frame}[fragile]{Ejemplo de Percentiles}
\begin{exampleblock}{✨ Ejemplo}
En un examen obtuviste 85 puntos
\end{exampleblock}

\begin{itemize}
  \item El P₉₀ es 80 puntos
\end{itemize}

\begin{itemize}
  \item Interpretación:
\end{itemize}

\begin{itemize}
  \item ✅ Superaste al 90% de estudiantes
\end{itemize}

\begin{itemize}
  \item ✅ Solo el 10% sacó más que tú
\end{itemize}

\begin{alertblock}{💡 Nota importante}
¡Excelente resultado! Estás en el top 10% 🌟
\end{alertblock}

\end{frame}

\begin{frame}[fragile]{El Diagrama de Cajón (Box Plot)}
\begin{itemize}
  \item Herramienta visual para ver distribución
\end{itemize}

\begin{alertblock}{💡 Nota importante}
¡Es como un resumen gráfico de 5 números! 📦
\end{alertblock}

\begin{itemize}
  \item Muestra de un vistazo:
\end{itemize}

\begin{itemize}
  \item 🔹 El mínimo
\end{itemize}

\begin{itemize}
  \item 🔹 Q1, Q2 (mediana), Q3
\end{itemize}

\begin{itemize}
  \item 🔹 El máximo
\end{itemize}

\end{frame}

\begin{frame}[fragile]{Partes del Diagrama de Cajón}
\begin{itemize}
  \item 🔵 Caja: va de Q1 a Q3
\end{itemize}

\begin{itemize}
  \item 📏 Rango intercuartílico (RIC) = Q3 - Q1
\end{itemize}

\begin{itemize}
  \item ━ Línea en la caja: la mediana (Q2)
\end{itemize}

\begin{itemize}
  \item 🔺 Bigotes: van desde la caja hasta mín y máx
\end{itemize}

\begin{alertblock}{💡 Nota importante}
¡La caja contiene el 50% central de los datos!
\end{alertblock}

\end{frame}

\begin{frame}[fragile]{Dibujando un Box Plot}
\begin{exampleblock}{🎯 Problema}
Datos: 5, 7, 8, 9, 10, 12, 15, 18, 20
\end{exampleblock}

\begin{center}
\colorbox{colorFondo}{%
  \parbox{0.9\textwidth}{%
    \begin{align*}
    Mínimo = 5, Máximo = 20
    \end{align*}
  }%
}
\end{center}

\begin{center}
\colorbox{colorFondo}{%
  \parbox{0.9\textwidth}{%
    \begin{align*}
    Q1 = 8, Q2 = 10, Q3 = 15
    \end{align*}
  }%
}
\end{center}

\begin{itemize}
  \item Bigote izquierdo: de 5 a 8
\end{itemize}

\begin{itemize}
  \item Caja: de 8 a 15 (con línea en 10)
\end{itemize}

\begin{itemize}
  \item Bigote derecho: de 15 a 20
\end{itemize}

\end{frame}

\begin{frame}[fragile]{Valores Atípicos (Outliers)}
\begin{itemize}
  \item Datos muy alejados del resto
\end{itemize}

\begin{alertblock}{💡 Nota importante}
¡Como alguien muy alto en una clase de niños pequeños! 🦒
\end{alertblock}

\begin{itemize}
  \item Se detectan con la regla:
\end{itemize}

\begin{center}
\colorbox{colorFondo}{%
  \parbox{0.8\textwidth}{%
    \begin{center}
    \Large\color{colorPrimario}
    $RIC = Q3 - Q1$
    \end{center}
  }%
}
\end{center}

\begin{itemize}
  \item Atípico si es < Q1 - 1.5×RIC
\end{itemize}

\begin{itemize}
  \item o si es > Q3 + 1.5×RIC
\end{itemize}

\end{frame}

\begin{frame}[fragile]{Ejemplo de Valor Atípico}
\begin{exampleblock}{✨ Ejemplo}
Datos: 10, 12, 11, 13, 12, 50
\end{exampleblock}

\begin{center}
\colorbox{colorFondo}{%
  \parbox{0.9\textwidth}{%
    \begin{align*}
    Q1=11, Q3=13, RIC=13-11=2
    \end{align*}
  }%
}
\end{center}

\begin{center}
\colorbox{colorFondo}{%
  \parbox{0.9\textwidth}{%
    \begin{align*}
    Límite superior: 13 + 1.5×2 = 16
    \end{align*}
  }%
}
\end{center}

\begin{itemize}
  \item El valor 50 > 16
\end{itemize}

\begin{alertblock}{💡 Nota importante}
¡50 es un valor atípico! Se marca con un punto especial ⚫
\end{alertblock}

\end{frame}

\begin{frame}[fragile]{Aplicación Práctica - Salarios}
\begin{exampleblock}{🎯 Problema}
Salarios mensuales (S/.): 1200, 1500, 1800, 2000, 2200, 2500, 3000, 3500, 5000
\end{exampleblock}

\begin{itemize}
  \item Paso 1: Ya están ordenados ✓
\end{itemize}

\begin{itemize}
  \item Paso 2: Calcular cuartiles
\end{itemize}

\begin{itemize}
  \item Paso 3: Hacer diagrama de cajón
\end{itemize}

\begin{itemize}
  \item Paso 4: Identificar atípicos
\end{itemize}

\end{frame}

\begin{frame}[fragile]{Solución - Cuartiles}
\begin{center}
\colorbox{colorFondo}{%
  \parbox{0.9\textwidth}{%
    \begin{align*}
    Mínimo = 1200, Máximo = 5000
    \end{align*}
  }%
}
\end{center}

\begin{center}
\colorbox{colorFondo}{%
  \parbox{0.9\textwidth}{%
    \begin{align*}
    Q2 (mediana) = 2200 (valor central)
    \end{align*}
  }%
}
\end{center}

\begin{itemize}
  \item Mitad baja: 1200, 1500, 1800, 2000
\end{itemize}

\begin{center}
\colorbox{colorFondo}{%
  \parbox{0.9\textwidth}{%
    \begin{align*}
    Q1 = (1500+1800)÷2 = 1650
    \end{align*}
  }%
}
\end{center}

\begin{itemize}
  \item Mitad alta: 2500, 3000, 3500, 5000
\end{itemize}

\begin{center}
\colorbox{colorFondo}{%
  \parbox{0.9\textwidth}{%
    \begin{align*}
    Q3 = (3000+3500)÷2 = 3250
    \end{align*}
  }%
}
\end{center}

\end{frame}

\begin{frame}[fragile]{Solución - Valores Atípicos}
\begin{center}
\colorbox{colorFondo}{%
  \parbox{0.9\textwidth}{%
    \begin{align*}
    RIC = 3250 - 1650 = 1600
    \end{align*}
  }%
}
\end{center}

\begin{center}
\colorbox{colorFondo}{%
  \parbox{0.9\textwidth}{%
    \begin{align*}
    Límite inferior: 1650 - 1.5×1600 = -750
    \end{align*}
  }%
}
\end{center}

\begin{center}
\colorbox{colorFondo}{%
  \parbox{0.9\textwidth}{%
    \begin{align*}
    Límite superior: 3250 + 1.5×1600 = 5650
    \end{align*}
  }%
}
\end{center}

\begin{itemize}
  \item Todos los salarios están entre -750 y 5650
\end{itemize}

\begin{alertblock}{💡 Nota importante}
¡No hay valores atípicos! Todos son normales
\end{alertblock}

\end{frame}

\begin{frame}[fragile]{Interpretación de Salarios}
\begin{itemize}
  \item 📊 50% de trabajadores gana entre 1650 y 3250
\end{itemize}

\begin{itemize}
  \item 📊 El salario típico (mediano) es 2200
\end{itemize}

\begin{itemize}
  \item 📊 25% gana menos de 1650
\end{itemize}

\begin{itemize}
  \item 📊 25% gana más de 3250
\end{itemize}

\begin{alertblock}{💡 Nota importante}
¡El box plot muestra que hay variabilidad en los salarios!
\end{alertblock}

\end{frame}

\begin{frame}[fragile]{Comparando con Box Plots}
\begin{itemize}
  \item Podemos comparar dos o más grupos
\end{itemize}

\begin{exampleblock}{✨ Ejemplo}
Comparar salarios de dos empresas lado a lado
\end{exampleblock}

\begin{itemize}
  \item Se puede ver de un vistazo:
\end{itemize}

\begin{itemize}
  \item ✓ Cuál tiene salarios más altos
\end{itemize}

\begin{itemize}
  \item ✓ Cuál tiene más variabilidad
\end{itemize}

\begin{itemize}
  \item ✓ Dónde hay valores atípicos
\end{itemize}

\begin{alertblock}{💡 Nota importante}
¡Es súper útil para comparaciones! 👥
\end{alertblock}

\end{frame}

\begin{frame}[fragile]{Ejercicio de Práctica}
\begin{exampleblock}{🎯 Problema}
Edades de un grupo: 15, 16, 15, 17, 16, 18, 15, 20, 16, 25
\end{exampleblock}

\begin{itemize}
  \item Tarea 1: Ordenar los datos
\end{itemize}

\begin{itemize}
  \item Tarea 2: Calcular Q1, Q2, Q3
\end{itemize}

\begin{itemize}
  \item Tarea 3: Calcular RIC
\end{itemize}

\begin{itemize}
  \item Tarea 4: Verificar si 25 es atípico
\end{itemize}

\begin{alertblock}{💡 Nota importante}
¡Intenta resolverlo antes de ver la solución!
\end{alertblock}

\end{frame}

\begin{frame}[fragile]{Solución del Ejercicio}
\begin{itemize}
  \item Ordenado: 15, 15, 15, 16, 16, 16, 17, 18, 20, 25
\end{itemize}

\begin{center}
\colorbox{colorFondo}{%
  \parbox{0.9\textwidth}{%
    \begin{align*}
    Q1 = 15, Q2 = 16, Q3 = 18
    \end{align*}
  }%
}
\end{center}

\begin{center}
\colorbox{colorFondo}{%
  \parbox{0.9\textwidth}{%
    \begin{align*}
    RIC = 18 - 15 = 3
    \end{align*}
  }%
}
\end{center}

\begin{center}
\colorbox{colorFondo}{%
  \parbox{0.9\textwidth}{%
    \begin{align*}
    Límite superior: 18 + 1.5×3 = 22.5
    \end{align*}
  }%
}
\end{center}

\begin{itemize}
  \item 25 > 22.5
\end{itemize}

\begin{alertblock}{💡 Nota importante}
¡Sí! 25 es un valor atípico (persona mayor en el grupo) ⚫
\end{alertblock}

\end{frame}

\begin{frame}[fragile]{¡Resumen de la Clase!}
\begin{itemize}
  \item ✅ Mediana: valor del centro (50%)
\end{itemize}

\begin{itemize}
  \item ✅ Cuartiles: dividen en 4 partes (Q1, Q2, Q3)
\end{itemize}

\begin{itemize}
  \item ✅ Percentiles: dividen en 100 partes
\end{itemize}

\begin{itemize}
  \item ✅ Box Plot: visualiza la distribución
\end{itemize}

\begin{itemize}
  \item ✅ Valores atípicos: datos muy alejados
\end{itemize}

\begin{alertblock}{💡 Nota importante}
¡Ahora puedes analizar dónde se ubican los datos! 🎯
\end{alertblock}

\end{frame}


% Diapositiva final atractiva
\begin{frame}[plain]
\begin{tikzpicture}[remember picture,overlay]
  \fill[colorAccento!20] (current page.south west) rectangle (current page.north east);
  \node[font=\Huge\bfseries,text=colorPrimario] at (current page.center) {¿Preguntas? 🤔};
  \node[font=\large,text=colorSecundario,below=1.5cm] at (current page.center) {¡Sigue aprendiendo! 🚀};
\end{tikzpicture}
\end{frame}

\end{document}