\documentclass[aspectratio=169]{beamer}
\usetheme{Madrid}
\usecolortheme{default}
\usepackage[utf8]{inputenc}
\usepackage[spanish]{babel}
\usepackage{amsmath}
\usepackage{amssymb}
\usepackage{tikz}
\usepackage{booktabs}

% Configuración del tema
\setbeamertemplate{navigation symbols}{}
\setbeamertemplate{footline}[frame number]

% Comandos personalizados
\newcommand{\alert}[1]{\textcolor{red}{#1}}
\newcommand{\highlight}[1]{\textcolor{blue}{#1}}

% Variables del documento (serán reemplazadas por Jinja2)
\title{Medidas de Posición}
\subtitle{Cuartiles, Percentiles y Diagrama de Cajón}
\author{Probabilidad y Estadística}
\date{\today}

\begin{document}

% Página de título
\begin{frame}
\titlepage
\end{frame}

% Diapositivas generadas dinámicamente
\begin{frame}{Cuartiles}
\begin{itemize}
\item Dividen datos ordenados en 4 partes iguales
\end{itemize}
\begin{itemize}
\item Q1: 25% de datos por debajo
\end{itemize}
\begin{itemize}
\item Q2: mediana (50%)
\end{itemize}
\begin{itemize}
\item Q3: 75% de datos por debajo
\end{itemize}
\begin{block}{Ejemplo}
Datos: 2, 4, 6, 8, 10, 12, 14, 16, 18
\end{block}
\begin{align*}
Q1=5, Q2=10, Q3=15
\end{align*}
\end{frame}

\begin{frame}{Percentiles}
\begin{itemize}
\item Dividen datos en 100 partes iguales
\end{itemize}
\begin{center}
\Large
$P_k: k\% de datos por debajo$
\end{center}
\begin{itemize}
\item P50 = Q2 = mediana
\end{itemize}
\begin{itemize}
\item P25 = Q1, P75 = Q3
\end{itemize}
\begin{block}{Ejemplo}
P90: 90% de datos están por debajo
\end{block}
\end{frame}

\begin{frame}{Diagrama de Cajón (Box Plot)}
\begin{itemize}
\item Representación visual de distribución
\end{itemize}
\begin{itemize}
\item Muestra: mínimo, Q1, Q2, Q3, máximo
\end{itemize}
\begin{itemize}
\item Identifica valores atípicos
\end{itemize}
\begin{itemize}
\item Caja: rango intercuartílico (Q1-Q3)
\item Línea central: mediana
\item Bigotes: extensión de datos
\end{itemize}
\begin{alertblock}{Nota}
Útil para comparar múltiples distribuciones
\end{alertblock}
\end{frame}

\begin{frame}{Aplicación Práctica}
\begin{exampleblock}{Problema}
Salarios mensuales (S/.): 1200, 1500, 1800, 2000, 2200, 2500, 3000, 3500, 5000
\end{exampleblock}
\begin{itemize}
\item Calcular Q1, Q2, Q3
\end{itemize}
\begin{itemize}
\item Construir diagrama de cajón
\end{itemize}
\begin{itemize}
\item Identificar valores atípicos
\end{itemize}
\begin{itemize}
\item Interpretar distribución salarial
\end{itemize}
\end{frame}


% Diapositiva final
\begin{frame}
\begin{center}
\Huge ¿Preguntas?
\end{center}
\end{frame}

\end{document}