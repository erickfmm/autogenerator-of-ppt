\documentclass[aspectratio=169]{beamer}
\usetheme{Madrid}
\usecolortheme{default}
\usepackage[utf8]{inputenc}
\usepackage[spanish]{babel}
\usepackage{amsmath}
\usepackage{amssymb}
\usepackage{tikz}
\usepackage{booktabs}

% Configuración del tema
\setbeamertemplate{navigation symbols}{}
\setbeamertemplate{footline}[frame number]

% Comandos personalizados
\newcommand{\alert}[1]{\textcolor{red}{#1}}
\newcommand{\highlight}[1]{\textcolor{blue}{#1}}

% Variables del documento (serán reemplazadas por Jinja2)
\title{Representación de datos a través de tablas y gráficos}
\subtitle{Organización y visualización de información}
\author{Probabilidad y Estadística}
\date{\today}

\begin{document}

% Página de título
\begin{frame}
\titlepage
\end{frame}

% Diapositivas generadas dinámicamente
\begin{frame}{Tablas de Frecuencia}
\begin{itemize}
\item Frecuencia absoluta: conteo de datos
\end{itemize}
\begin{itemize}
\item Frecuencia relativa: proporción del total
\end{itemize}
\begin{block}{Ejemplo}
Notas de estudiantes: 15, 18, 15, 20, 18, 15
\end{block}
\begin{center}
\begin{tabular}{|c|c|c|)}
\toprule
Nota & Frec. Abs. & Frec. Rel. \\
\midrule
15 & 3 & 0.50 \\
18 & 2 & 0.33 \\
20 & 1 & 0.17 \\
\bottomrule
\end{tabular}
\end{center}
\end{frame}

\begin{frame}{Tipos de Gráficos}
\begin{itemize}
\item Gráfico de barras: comparar categorías
\end{itemize}
\begin{itemize}
\item Histograma: distribución de datos continuos
\end{itemize}
\begin{itemize}
\item Gráfico circular: proporciones del todo
\end{itemize}
\begin{itemize}
\item Gráfico de líneas: tendencias en el tiempo
\end{itemize}
\begin{alertblock}{Nota}
Elegir según tipo de datos y objetivo
\end{alertblock}
\end{frame}

\begin{frame}{Promedio (Media Aritmética)}
\begin{center}
\Large
$\bar{x} = \frac{\sum x_i}{n}$
\end{center}
\begin{itemize}
\item Suma de valores dividida entre cantidad de datos
\end{itemize}
\begin{block}{Ejemplo}
Datos: 5, 8, 10, 12, 15
\end{block}
\begin{align*}
\bar{x} = \frac{5+8+10+12+15}{5} = \frac{50}{5} = 10
\end{align*}
\end{frame}

\begin{frame}{Ejercicio Aplicado}
\begin{exampleblock}{Problema}
Temperaturas semanales (°C): 22, 24, 21, 23, 25, 20, 22
\end{exampleblock}
\begin{itemize}
\item Construir tabla de frecuencias
\end{itemize}
\begin{itemize}
\item Calcular promedio
\end{itemize}
\begin{itemize}
\item Crear gráfico apropiado
\end{itemize}
\begin{alertblock}{Nota}
Interpretar resultados en contexto
\end{alertblock}
\end{frame}


% Diapositiva final
\begin{frame}
\begin{center}
\Huge ¿Preguntas?
\end{center}
\end{frame}

\end{document}