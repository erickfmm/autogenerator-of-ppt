\documentclass[aspectratio=169]{beamer}
\usetheme{Madrid}
\usecolortheme{default}
\usepackage[utf8]{inputenc}
\usepackage[spanish]{babel}
\usepackage{amsmath}
\usepackage{amssymb}
\usepackage{tikz}
\usepackage{booktabs}

% Configuración del tema
\setbeamertemplate{navigation symbols}{}
\setbeamertemplate{footline}[frame number]

% Comandos personalizados
\newcommand{\alert}[1]{\textcolor{red}{#1}}
\newcommand{\highlight}[1]{\textcolor{blue}{#1}}

% Variables del documento (serán reemplazadas por Jinja2)
\title{<< tema >>}
\subtitle{<< subtitulo >>}
\author{Probabilidad y Estadística}
\date{\today}

\begin{document}

% Página de título
\begin{frame}
\titlepage
\end{frame}

% Diapositivas generadas dinámicamente
<<% for slide in diapositivas %>>
\begin{frame}{<< slide.titulo >>}
<<% for item in slide.contenido %>>
<<% if item is string %>>
\begin{itemize}
\item << item >>
\end{itemize}
<<% elif item.tipo == "ejemplo" %>>
\begin{block}{Ejemplo}
<< item.texto >>
\end{block}
<<% elif item.tipo == "formula" %>>
\begin{center}
\Large
$<< item.texto >>$
\end{center}
<<% elif item.tipo == "calculo" %>>
\begin{align*}
<< item.texto >>
\end{align*}
<<% elif item.tipo == "nota" %>>
\begin{alertblock}{Nota}
<< item.texto >>
\end{alertblock}
<<% elif item.tipo == "problema" %>>
\begin{exampleblock}{Problema}
<< item.texto >>
\end{exampleblock}
<<% elif item.tipo == "tabla" %>>
\begin{center}
\begin{tabular}{<<% for h in item.encabezados %>>|c<<% endfor %>>|)}
\toprule
<<% for h in item.encabezados %>><< h >><<% if not loop.last %>> & <<% endif %>><<% endfor %>> \\
\midrule
<<% for fila in item.filas %>>
<<% for celda in fila %>><< celda >><<% if not loop.last %>> & <<% endif %>><<% endfor %>> \\
<<% endfor %>>
\bottomrule
\end{tabular}
\end{center}
<<% elif item.tipo == "componentes" %>>
\begin{itemize}
<<% for comp in item.lista %>>
\item << comp >>
<<% endfor %>>
\end{itemize}
<<% elif item.tipo == "solucion" %>>
\begin{block}{Solución}
\begin{itemize}
<<% for paso in item.pasos %>>
\item << paso >>
<<% endfor %>>
\end{itemize}
\end{block}
<<% endif %>>
<<% endfor %>>
\end{frame}

<<% endfor %>>

% Diapositiva final
\begin{frame}
\begin{center}
\Huge ¿Preguntas?
\end{center}
\end{frame}

\end{document}
